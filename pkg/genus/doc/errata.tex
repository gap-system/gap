%%%%%%%%%%%%%%%%%%%%%%%%%%%%%%%%%%%%%%%%%%%%%%%%%%%%%%%%%%%%%%%%%%%%%%%%%%%%%
%%
%W  errata.tex             GAP 4 package `genus'                Thomas Breuer
%%
%H  $Id: errata.tex,v 1.4 2002/08/21 15:03:34 gap Exp $
%%
%Y  Copyright (C) 2001,  Lehrstuhl D fuer Mathematik,   RWTH Aachen,  Germany
%%
%%  Errata et Addenda for the book.
%%
%%  Construct an HTML format file from this file using
%%  `tth -t -Lerrata < errata.tex > ../htm/errata.htm'.
%%
\documentclass[12pt,twoside]{article}

%%%%%%%%%%%%%%%%%%%%%%%%%%%%%%%%%%%%%%%%%%%%%%%%%%%%%%%%%%%%%%%%%%%%%%%%%%%%%
% from the file `rie.tex'
\textwidth138truemm
\textheight215truemm
\def\tthdump#1{#1}
\tthdump{\setlength{\evensidemargin}{\the\paperwidth}%
      \addtolength{\evensidemargin}{-2truein}%
      \addtolength{\evensidemargin}{-2\the\hoffset}%
      \addtolength{\evensidemargin}{-\the\oddsidemargin}%
      \addtolength{\evensidemargin}{-\the\textwidth}}

%%%%%%%%%%%%%%%%%%%%%%%%%%%%%%%%%%%%%%%%%%%%%%%%%%%%%%%%%%%%%%%%%%%%%%%%%%%%%
\def\refname{\centerline{\large\bf References}}


%%%%%%%%%%%%%%%%%%%%%%%%%%%%%%%%%%%%%%%%%%%%%%%%%%%%%%%%%%%%%%%%%%%%%%%%%%%%%
% Miscellaneous macros.
\def\GAP{\textsf{GAP}}
\def\ATLAS{\textsf{ATLAS}}
%%tth: \font\Bbb=msbm10
\def\N{{\Bbb N}} \def\Z{{\Bbb Z}} \def\Q{{\Bbb Q}} \def\R{{\Bbb R}}
\def\C{{\Bbb C}} \def\F{{\Bbb F}}
\def\Span#1{\langle #1 \rangle}
\def\Spur{\textrm{Tr}}

%%%%%%%%%%%%%%%%%%%%%%%%%%%%%%%%%%%%%%%%%%%%%%%%%%%%%%%%%%%%%%%%%%%%%%%%%%%%%
\tthdump{\vfuzz=2pt}

%%%%%%%%%%%%%%%%%%%%%%%%%%%%%%%%%%%%%%%%%%%%%%%%%%%%%%%%%%%%%%%%%%%%%%%%%%%%%
\begin{document}

\parskip3mm
\parindent0pt

%%tth: \begin{html} <body bgcolor="FFFFFF"> \end{html}
\tthdump{\vspace*{-2cm}}

\begin{center}
%%tth: \title{Errata et Addenda}
\tthdump{\large Errata et Addenda} \\[0.5cm]
{\normalsize for} \\[0.5cm]
%%tth: \begin{html} <br /> \end{html}
{\large Characters and Automorphism Groups \\
of Compact Riemann Surfaces} \\[0.5cm]
%%tth: \begin{html} <br /> \end{html}
{\normalsize by} \\[0.5cm]
%%tth: \begin{html} <br /> \end{html}
{\large Thomas Breuer \\[0.5cm]
%%tth: \begin{html} <br /> \end{html}
{\normalsize
\textit{Lehrstuhl D f{\"u}r Mathematik} \\
\textit{RWTH, 52056 Aachen, Germany} \\
E-mail: sam@math.rwth-aachen.de}}
%E-mail: sam@math.rwth-aachen.de \\[0.5cm]
%%%tth: \begin{html} <br /> \end{html}
%September 9th, 2001}}
\end{center}


%%%%%%%%%%%%%%%%%%%%%%%%%%%%%%%%%%%%%%%%%%%%%%%%%%%%%%%%%%%%%%%%%%%%%%%%%%%%%
\vspace*{1cm}
\centerline{\large\bf Errata}

\begin{description}
\item[p.~20, l.~-4 to -1:]
    Theorems~3A and~3B in~\cite{Gre63} are not correct,
    so replace the theorem by the following:

    {\scshape Theorem}~5.1 (\cite{Sin72}, Theorems~1~and~2]).
    {\itshape 
    No Fuchsian group with signature $(g;m_1,m_2,\ldots,m_r)$ is finitely
    maximal if and only if the signature is one of
    \[
       (0;2,n,2n), (0;3,n,3n), (0;m,m,n), (0;m,m,n,n), (1;n,n), (1;n),
       (2;-) .
    \]}
    %\vspace*{-1cm}
    % (found by myself on August 19th, 2001)
    (The error does not affect the material in the rest of the book.)
    % (added in March 2002, after a question of G. Gromadzki)

\item[p.~52, l.~6:]
    Insert ``at least'' after ``has''.
    % (found by myself in 2000)

\item[p.~66, l.~-9 to -6:]
    The formulation of Lemma~17.6 may be misleading,
    so replace it by the following:

    {\scshape Lemma}~17.6.
    {\itshape 
    Let $\Gamma$ be a Fuchsian group, and $m$ a positive integer such that
    no group of order $m$ is perfect.
    If $\Gamma$ has a surface kernel factor of order $m$ then there is
    a prime $p$ dividing} gcd$(m, [\Gamma : \Gamma^{\prime}])$
    {\itshape
    and a normal subgroup of index $p$ in $\Gamma$ whose signature
    is admissible for $m/p$.}
    % (brought up by A. Wootton in Dec. 2001)

\item[p.~80, l.~-11:]
    Add a closing bracket after {\tt 9E}.
    % (found by myself in 2000)

\item[p.~100, l.~-1 and p.~112, l.~3:]
    $\textrm{Write } \Spur(\Phi) + \overline{\Spur(\Phi)}
    \textrm{ instead of } \Phi + \overline{\Phi}$.
    % (found by myself on August 19th, 2001)

\item[p.~118, l.~-11 to -9:]
    Remove this paragraph.
    % (found by myself in 2000)

\item[p.~169, l.~7:]
    Add that $p \geq 7$ is required,
    since for $p = 3$, $\chi$ is not a proper character.
    (Note that Lemma~34.8 holds also for $p=3$.)
    % (found by myself on August 20th, 2001)

\item[p.~189, l.~-1:]
    $\textrm{Write det}( D_m ) = 0$.
    % (found by myself on May 25th, 2002)

\item[p.~194, l.~-14:]
    Add the reference~\cite{Sin72},
    which (after the above correction) is referred to on p.~20.
    % (found by myself on August 19th, 2001)
\end{description}


%%%%%%%%%%%%%%%%%%%%%%%%%%%%%%%%%%%%%%%%%%%%%%%%%%%%%%%%%%%%%%%%%%%%%%%%%%%%%
\vspace*{1cm}
\tthdump{\newpage}
\centerline{\large\bf Addenda}

\begin{description}
\item[p.~62 f.:]
    The $({\tt 2C}, {\tt 3D}, {\tt 8C})$-generation of the group $Fi_{23}$
    established in Section~16 with character-theoretic methods
    has been proved by Robert A.~Wilson,
    via explicit computations with the group $Fi_{23}$.

    He has computed also the (strong) symmetric genera of the Baby Monster
    and the Monster.
    For the Baby Monster, it arises from $(2,3,8)$-generation.
    The Monster is a Hurwitz group.

    For details, see~\cite{Wil93,Wil97,Wil01}.
    % (found by myself in 2000)

\item[p.~98, l.~5 to 7:]
    For a character that comes from a Riemann surface,
    the representation of the sum with its complex conjugate
    in terms of permutation characters
    has been derived also by A.~J.~Broughton;
    in~\cite{Bro90}, this is used to prove Corollary~15.10 in an
    alternative way,
    which can be rephrased in our terminology, as follows.

    Suppose that the elements $x_1$, $x_2$, \ldots, $x_r$ with the
    property $x_1 x_2 \cdots x_r = 1$ generate the group $G$.
    This gives rise to a surface kernel epimorphism
    $\Phi : \Gamma(0;|x_1|, |x_2|, \ldots, |x_r|) \rightarrow G$,
    with induced character $\Spur(\Phi)$.
    By Corollary~22.5, we have
    $$\Spur(\Phi) + \overline{\Spur(\Phi)} = 2 \cdot 1_G - 2 \cdot \rho_G
    + \sum_{i=1}^r ( \rho_G - 1_{\Span{x_i}}^G ) .$$
    For any character $\chi$ of $G$, the scalar product with this character
    is clearly nonnegative, thus
    $$2 \cdot [ \chi , \rho_G - 1_G ]
    \leq \sum_{i=1}^r [ \chi , \rho_G - 1_{\Span{x_i}}^G ] .$$
    Because of $[ \chi , \rho_G ] = \chi(1)$ and together with
    Frobenius reciprocity, this implies
    $$2 \cdot ( \chi(1) - [ \chi , 1_G ] )
    \leq \sum_{i=1}^r ( \chi(1) - [ \chi_{\Span{x_i}} , 1_{\Span{x_i}} ] ) .$$

    (In~\cite{Bro90}, this is in fact stated also for the case that
    the preimage of $\Phi$ has positive orbit genus.
    But then the analogon of the above condition is trivially satisfied.)
    % (this fact from the paper noticed by myself on August 13th, 2002)
\end{description}

% add a note on A. J. Broughton's paper!~\cite{Bro90}
% chapter 4: "finer" classification than characters,
% loop up to conjugacy


%%%%%%%%%%%%%%%%%%%%%%%%%%%%%%%%%%%%%%%%%%%%%%%%%%%%%%%%%%%%%%%%%%%%%%%%%%%%%
\bibliographystyle{amsalpha}
\bibliography{../../../doc/mrabbrev,../../../doc/manual,manual}

%%tth: \begin{html} <hr> \end{html}
Last update August~20th, 2002.
\end{document}

%%%%%%%%%%%%%%%%%%%%%%%%%%%%%%%%%%%%%%%%%%%%%%%%%%%%%%%%%%%%%%%%%%%%%%%%%%%%%
%%
%E

