%%%%%%%%%%%%%%%%%%%%%%%%%%%%%%%%%%%%%%%%%%%%%%%%%%%%%%%%%%%%%%%%%%%%%%%%%
%%
%W  install.tex            GAP documentation                Heiko Dietrich
%%
%H  $Id: install.tex,v 1.4 2007/05/08 07:51:11 gap Exp $
%%

%%%%%%%%%%%%%%%%%%%%%%%%%%%%%%%%%%%%%%%%%%%%%%%%%%%%%%%%%%%%%%%%%%%%%%%%%
\Chapter{Installing and loading the Cubefree package}

\atindex{Installing and loading the Cubefree package}{@installing %
                        and loading the {\Cubefree} package}
%%%%%%%%%%%%%%%%%%%%%%%%%%%%%%%%%%%%%%%%%%%%%%%%%%%%%%%%%%%%%%%%%%%%%%%%%
\Section{Installing the Cubefree package}\null

\atindex{Installing the Cubefree package}{@installing the {\Cubefree} package}
The installation of the {\Cubefree} package follows standard {\GAP} rules.
So the standard method is to unpack the package into the `pkg'
directory  of your {\GAP} distribution.  This will create an `cubefree'
subdirectory. 

%%%%%%%%%%%%%%%%%%%%%%%%%%%%%%%%%%%%%%%%%%%%%%%%%%%%%%%%%%%%%%%%%%%%%%%%%
\Section{Loading the Cubefree package}\null

\atindex{Loading the Cubefree package}{@loading the {\Cubefree} package}
To use the {\Cubefree} Package you have to request it explicitly. This  is
done by calling `RequirePackage' like this:

\beginexample
gap> RequirePackage("Cubefree");
Loading Cubefree 1.05 ...

   -- Construction Algorithm for Cubefree Groups, 1.05 --
   -------- Heiko Dietrich, H.Dietrich@tu-bs.de ---------
true
\endexample


%%%%%%%%%%%%%%%%%%%%%%%%%%%%%%%%%%%%%%%%%%%%%%%%%%%%%%%%%%%%%%%%%%%%%%%%%
%%
%E
