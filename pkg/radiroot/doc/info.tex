%%%%%%%%%%%%%%%%%%%%%%%%%%%%%%%%%%%%%%%%%%%%%%%%%%%%%%%%%%%%%%%%%%%%%%%%%
%%
%W  info.tex          Radiroot documentation             Andreas Distler
%%
%H  $Id: info.tex,v 1.2 2006/10/30 14:44:55 gap Exp $
%%
%Y  2006
%%

%%%%%%%%%%%%%%%%%%%%%%%%%%%%%%%%%%%%%%%%%%%%%%%%%%%%%%%%%%%%%%%%%%%%%%%%%
\Chapter{The Info Class of the Package}

The `info' mechanism in {\GAP} allows functions to print information
during the computation (see Section ~"ref:Info Functions" in the
{\GAP} reference manual for general information).

\> InfoRadiroot

is the info class of this package.

\> SetInfoLevel( InfoRadiroot, <level> )

sets the info level for `InfoRadiroot' to <level>, where <level> has
to be an integer in the range 0-4.

The default value for `InfoRadiroot' is 1. Information why a function
returns `fail' will be given with this setting.
\beginexample
gap> InfoLevel(InfoRadiroot);
1
gap> RootsOfPolynomialAsRadicals(x^5-4*x+2);
#I  Polynomial is not solvable.
fail
\endexample

Setting the info level to a higher value will cause messages to show up during
single steps of the computation. On level 2 one gets a rough
overview. Those who want to go into the details of the algorithm
described in \cite{Distler05} and of the implementation itself will
find the information on level 3-4 helpful.

To use the package in silent mode the info level can be given the
value 0.
