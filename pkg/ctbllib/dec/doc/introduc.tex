%%%%%%%%%%%%%%%%%%%%%%%%%%%%%%%%%%%%%%%%%%%%%%%%%%%%%%%%%%%%%%%%%%%%%%%%%
%%
%W  introduc.tex              GAP applications              Thomas Breuer
%%
%H  @(#)$Id: introduc.tex,v 1.2 2010/01/27 15:43:41 gap Exp $
%%
%Y  Copyright 1999,  Lehrstuhl D fuer Mathematik,  RWTH Aachen,   Germany
%%
%X  NAME="introduc"
%X  chmod 444 $NAME.tex 
%X  latex $NAME; latex $NAME
%X  pdflatex $NAME
%X  tth -u -L$NAME < $NAME.tex > ../htm/$NAME.htm
%%
\documentclass[a4paper]{article}

\textwidth16cm
\oddsidemargin0pt

\parskip 1ex plus 0.5ex minus 0.5ex
\parindent0pt

\usepackage{amssymb}

% Miscellaneous macros.
\def\GAP{\textsf{GAP}}
\def\ATLAS{\textsc{ATLAS}}
\def\N{{\mathbb N}} \def\Z{{\mathbb Z}} \def\Q{{\mathbb Q}}
\def\R{{\mathbb R}} \def\C{{\mathbb C}} \def\F{{\mathbb F}}
\def\tthdump#1{#1}
\tthdump{\def\URL#1{\texttt{#1}}}
%%tth: \def\URL#1{\url{#1}}
%%tth: \def\abstract#1{#1}

\begin{document}

\title{How to Read the Decomposition Matrices}

\author{\textsc{Thomas Breuer} \\[0.5cm]
\textit{Lehrstuhl D f{\"u}r Mathematik} \\
\textit{RWTH, 52056 Aachen, Germany}}

\date{July 18th, 1999}


%%%%%%%%%%%%%%%%%%%%%%%%%%%%%%%%%%%%%%%%%%%%%%%%%%%%%%%%%%%%%%%%%%%%%%%%%
\maketitle


%%%%%%%%%%%%%%%%%%%%%%%%%%%%%%%%%%%%%%%%%%%%%%%%%%%%%%%%%%%%%%%%%%%%%%%%%

For a fixed characteristic $p$ and a fixed simple group $G$,
various bicyclic extensions $m.G.a$ may occur,
where $G.a$ is a subgroup of the automorphism group of $G$
and $m$ is a cyclic factor of the Schur multiplier of $G.a$.
For each such group $G.a$, the information about all bicyclic extensions
$m.G.a$ that occur in the {\ATLAS} of Finite Groups is shown together
in one document.

% Note that for cases such as $A_6.2_3$, no decomposition matrices
% are shown for $4.A_6.2_3$.

Each file starts with the {\bf name} of the group $G.a$ and the
characteristic $p$,
followed by an {\bf overview} of the blocks.
Finally, the {\bf decomposition matrices} themselves are listed blockwise.


%%%%%%%%%%%%%%%%%%%%%%%%%%%%%%%%%%%%%%%%%%%%%%%%%%%%%%%%%%%%%%%%%%%%%%%%%
\section{The Blocks Overview}

The blocks overview is a four-column table which is divided into several
portions of rows.
Each portion of rows describes the blocks containing faithful
ordinary irreducible characters of a group of structure $m.G.a$,
and this is denoted by the entry $m.G$ in the first column.
The second column lists a number for each block,
this number occurs in the upper left corner of the decomposition matrices
shown further down.
The third column lists the defects of the blocks.
In the fourth column, for blocks of nonzero defect the dimensions of the
decomposition matrices are listed;
for each block of defect zero, the $1 \times 1$ decomposition matrix
is omitted later on,
instead the degree and the labels of the ordinary and the Brauer
character in the block are listed in the overview part of the file.

The blocks are ordered according to the appearance of ordinary
irreducibles.
That is, the first ordinary character in the $i$-th block appears later
than the first in the $(i-1)$-th block and earlier than the first in the
$(i+1)$-th block.
Thus the principal block is always the first one.

For central extensions $m.G.a$ with $m \geq 3$,
the {\ATLAS} prints only those faithful characters for which
the restriction to the centre of $m.G$ is a multiple of a particular
irreducible character.
In these cases, the decomposition matrix of a faithful block is shown
only if its ordinary characters are printed in the {\ATLAS}.
The conjugate blocks consisting of not printed characters are then the
following blocks, their entries in columns three and four of the table
are omitted, and their block numbers in the second column are extended
by a description which decomposition matrix is taken instead.
The following descriptions occur.
$j = i\ast$ means that block $j$ is the unique block consisting of
characters for which the printed ones lie in the $i$-th block,
$j = i\ast k$ means that block $j$ consists of the images of the
characters in the $i$-th block under the algebraic conjugacy $\ast k$,
and $j = \overline{i}$ means that block $j$ is the complex conjugate of
the $i$-th block.

The blocks of a specific group $m.G.a$ are read off as follows.
Let $M$ be the order of a maximal cyclic factor of the Schur multiplier
of the simple group $G$ that is contained in the {\ATLAS} of Finite
Groups,
and $m.G.a$ a bicyclic extension where $m$ divides $M$.

If $M$ is coprime to the characteristic $p$ then the set of blocks of
$m.G.a$ is the union of blocks corresponding to the row portions
of the factor groups of $m.G.a$,
in the same way as the set of irreducible characters of $m.G.a$ is given
in the {\ATLAS} of Finite Groups and the {\ATLAS} of Brauer Characters.
For example, the $5$-blocks of $2.A_6$ are the blocks for $G$ and $2.G$,
the $5$-blocks of $3.A_6$ are the blocks for $G$ and $3.G$,
and the $5$-blocks of $6.A_6$ are the blocks for $G$, $2.G$, $3.G$, and
$6.G$.

If $p$ divides $M$ then either $p = 2$ or $p = 3$,
since $M$ divides $12$ for all {\ATLAS} groups.
In such a case, the first column of the overview table contains values
$m.G$ only for $m$ divisible by the $p$-part of $M$.
For groups $m.G.a$ where $m$ has this property the set of blocks is again
the union of blocks of the appropriate factor groups.
In the general case, let $m^{\prime}$ be the least common multiple of $m$
and the $p$-part of $M$.
Then the blocks of $m.G.a$ are in bijection with the blocks of
$m^{\prime}.G.a$, the sets of Brauer characters of the two groups
coincide,
and the set of ordinary characters in each block of $m.G.a$ is a subset
of the ordinary characters in the corresponding block of
$m^{\prime}.G.a$;
in the decomposition matrices shown later, the portions of ordinary
irreducibles in the block for the factor groups of $m^{\prime}.G.a$ by
$p$-groups are separated by horizontal lines.
For example, the $2$-blocks of $6.A_6$ are the blocks in the portions
for $2.G$ and $6.G$, the $2$-blocks of $A_6$ are the restrictions of the
blocks of $2.A_6$ to the simple group.

Note that due to this notation, the defects of blocks of factor groups
may be smaller than the defects listed in the third column of the
overview table.
For example, in characteristic $3$ the group $A_6$ has a character
$\chi_6$ of degree $9$ which hence is of defect zero.
But viewed as a character of $3.G$, it occurs in a block of defect one,
together with the character $\chi_{17}$ and its complex conjugate.

In the case $m \not= m^{\prime}$, one more phenomenon may occur.
Namely, if the centres of $m.G.a$ and $m.G$ differ then additionally
the characters in a block of $m^{\prime}.G.a$ may lie in different blocks
of $m.G.a$.
For {\ATLAS} groups, this actually happens only for the case that an
outer automorphism of $G$ acts on $m.G$ by inverting the elements in the
centre,
and the restriction of the shown block of $m^{\prime}.G.a$ to $m.G.a$
splits into two blocks of this group.
This phenomenon is indicated by a vertical and horizontal line in the
decomposition matrix of the block of $m^{\prime}.G.a$.

For example, consider the group $6.A_7.2$ in characteristic $3$.
The unique defect $2$ block of $3.A_7.2$ contains three faithful ordinary
irreducible characters, six odinary irreducibles of the actor group
$A_7.2$, and four Brauer characters of $A_7.2$.
The restriction to $A_7.2$ consists of six ordinary and four Brauer
characters, lying in two blocks of $A_7.2$.
The same happens also for the unique defect $1$ block of $6.A_7.2$.


%%%%%%%%%%%%%%%%%%%%%%%%%%%%%%%%%%%%%%%%%%%%%%%%%%%%%%%%%%%%%%%%%%%%%%%%%
\section{Format of the Decomposition Matrices}

In the upper left corner of each decomposition matrix,
the number of the block is shown.

The rows of a decomposition matrix correspond to ordinary characters,
the columns to Brauer characters.
These characters are denoted by labels
that refer to the labels used for the ordinary irreducibles and
irreducible Brauer characters in the {\ATLAS} of Finite Groups
and the {\ATLAS} of Brauer Characters, respectively.
Additionally, for each ordinary character its degree is given,
followed by the number $i$ such that the character is the $i$-th
irreducible character of this degree.
The degrees of the irreducible Brauer characters of each block are
listed in a table that follows the decomposition matrix of this block.

If a decomposition matrix does not fit on one page then the labels are
repeated for all parts of the matrix.
The block number in the second etc.~part of the matrix is written in
brackets.

Each zero entry of the matrix is represented by a dot.

The labels of the characters have the following form.
We state the rules only for ordinary characters,
the rules for Brauer characters are obtained by replacing $\chi$
by $\varphi$.

First consider only downward extensions $m.G$ of a simple group $G$.
If $m \leq 2$ then only labels of the form $\chi_i$ and $\varphi_j$
occur,
which denote the $i$-th ordinary and the $j$-th Brauer character shown in
the {\ATLAS}.
The labels of faithful ordinary characters of groups $m.G$ with $m\geq 3$
are of the form $\chi_i$, $\chi_i^{\ast}$, or $\chi_i^{\ast k}$,
which means the $i$-th character printed in the {\ATLAS},
the unique character that is not printed and for which $\chi_i$ acts as
proxy
(see~Sections~8 and~19 of Chapter~7 in the {\ATLAS} of Finite Groups),
and the image of the printed character $\chi_i$ under the algebraic
conjugacy operator $\ast k$, respectively.

For groups $m.G.a$ with $a > 1$, the labels of the irreducible characters
are derived from the labels of the irreducible constituents of their
restrictions to $m.G$, as follows.
\begin{enumerate}
\item
    If the ordinary irreducible character $\chi_i$ of $m.G$ extends to
    $m.G.a$ then the $a^{\prime}$ extensions are denoted by
    $\chi_{i,0}$, $\chi_{i,1}$, \ldots, $\chi_{i,a^{\prime}}$,
    where $\chi_{i,0}$ is the character whose values are printed in the
    {\ATLAS}.
\item
    The label $\chi_{i_1+}$ means that $a$ different characters
    $\chi_{i_1}$, $\chi_{i_2}$, \ldots, $\chi_{i_a}$ of $m.G$ induce to
    an irreducible character of $m.G.a$ with this label.
    Note that $i_2$, $i_3$, \ldots, $i_a$ can be read off from the
    fusion signs in the {\ATLAS}.
\item
    Finally, the label $\chi_{i,j+}$ means that the character $\chi_i$
    of $m.G$ extends to a group that lies properly between $m.G$ and
    $m.G.a$, and the extension $\chi_{i,j}$ induces to $\chi_{i,j+}$.
\end{enumerate}

Horizontal lines in a block of a maximal cyclic downward extension
$m.G.a$ of $G.a$ are used to separate portions of ordinary characters
whose restrictions to the centre of $m.G$ are multiples of different
irreducible characters.
This happens only if the characteristic $p$ is either $2$ or $3$ and
$p$ divides $m$.
In this situation, a vertical line signals that the restriction of the
block to a factor group by a $p$-group splits into two blocks.
Examples are provided by the $3$-blocks of $6.A_7$ and $6.A_7.2$.


%%%%%%%%%%%%%%%%%%%%%%%%%%%%%%%%%%%%%%%%%%%%%%%%%%%%%%%%%%%%%%%%%%%%%%%%%
\section{The Special Cases $L_3(4)$ and $U_4(3)$}

The groups $L_3(4)$ and $U_4(3)$ need further explanations because each
of them has two nonisomorphic downward extensions by a cyclic group of
order $12$.

In the case of $G = L_3(4)$ in characteristic $2$,
the restrictions of the ordinary characters of the two nonisomorphic
groups $4_1.G.a$ and $4_1.G.a$ to $2$-regular classes are Brauer
characters of the group $G.a$,
and the ordinary characters of both
$12_1.G.a$ and $12_1.G.a$ restrict to Brauer characters of $3.G.a$.
We decided to print and count the blocks of $12_1.G.a$ and $12_2.G.a$
in characteristic $2$ separately since it makes no sense to list ordinary
characters in one block which do not occur in a common group.
Note that this means that the part of $6.G.a$ is listed twice.

Analogously, in the case of $G = U_4(3)$ and characteristic $3$,
the decomposition matrices of $12_1.G.a$ and $12_2.G.a$ are shown
separately,
thus listing the part of $4.G.a$ twice.


%%%%%%%%%%%%%%%%%%%%%%%%%%%%%%%%%%%%%%%%%%%%%%%%%%%%%%%%%%%%%%%%%%%%%%%%%
\end{document}


%%%%%%%%%%%%%%%%%%%%%%%%%%%%%%%%%%%%%%%%%%%%%%%%%%%%%%%%%%%%%%%%%%%%%%%%%
%%
%E

