%%%%%%%%%%%%%%%%%%%%%%%%%%%%%%%%%%%%%%%%%%%%%%%%%%%%%%%%%%%%%%%%%%%%%%%%%%%%
%%
%W  lpres.tex		NQL Doc				Rene Hartung
%%
%H  $Id: lpres.tex,v 1.15 2009/05/06 13:26:26 gap Exp $
%%

%%%%%%%%%%%%%%%%%%%%%%%%%%%%%%%%%%%%%%%%%%%%%%%%%%%%%%%%%%%%%%%%%%%%%%%%%%%%
\Chapter{An Introduction to L-presented groups}

Let $S$ be an alphabet, $Q$ and $R$ be subsets of the free group $F_S$
over this alphabet, and $\Phi$ be a set of free group endomorphisms
$\varphi\colon F_S\to F_S$. An <$L$-presentation> is a quadruple
$(S,Q,\Phi,R)$ and it is called <finite> if the
sets $S$, $Q$, $\Phi$, and $R$ are finite. A (finite) $L$-presentation
$(S,Q,\Phi,R)$ defines the (<finitely>) <$L$-presented group>
$$ G=\left\langle S~\left|~ Q\cup \bigcup_{\varphi\in\Phi^\*}
   R^\varphi\right.\right\rangle $$
where $\Phi^\*$ denotes the free monoid generated by $\Phi$;
that is, the closure of $\Phi\cup\{{\rm id}\}$ under composition.

The elements in $Q$ are the <fixed relators> and the elements in $R$
are the <iterated relators> of the $L$-presentation $(S,Q,\Phi,R)$. An
$L$-presentation of the form $(S,\emptyset,\Phi,R)$ is an <ascending
$L$-presentation> and it is an <invariant $L$-presentation> if the
normal subgroup
$$ K=\left\langle Q\cup \bigcup_{\varphi\in\Phi^\*}
   R^\varphi\right\rangle^{F_S} $$
is $\varphi$-invariant for each $\varphi\in\Phi$; that is, if $K$
satisfies $K^\varphi\subset K$ for each $\varphi\in\Phi$.  Note that every
ascending $L$-presentation is invariant and for each $L$-presentation
$(S,Q,\Phi,R)$ there is a unique <underlying ascending $L$-presentation>
$(S,\emptyset,\Phi,R)$ which is invariant. In general it is not decidable
whether or not a given $L$-presentation is invariant as this would
require a solution to the word-problem.

In the remainder of this manual, an $L$-presented group is always
finitely $L$-presented.

%%%%%%%%%%%%%%%%%%%%%%%%%%%%%%%%%%%%%%%%%%%%%%%%%%%%%%%%%%%%%%%%%%%%%%%%%%%%
\Section{Creating an L-presented group}

The construction of an $L$-presented group is similar to the construction of a
finitely presented group (see Chapter~"ref:Creating Finitely Presented Groups" 
of the \GAP\ Reference manual for further details).

\> LPresentedGroup( <F>, <frels>, <endos>, <irels> ) F

returns the {\GAP} object of an $L$-presented group with the underlying
free group <F>, the fixed relators <frels>, the set of endomorphisms
<endos>, and the iterated relators <irels>. The input variables <frels>
and <irels> need to be finite subsets of the underlying free group <F>
and <endos> needs to be a finite list of homomorphisms $F\to F$.

For example, the Grigorchuk group, 
$$ \Big\langle a,b,c,d~\Big|~a^2,b^2,c^2,d^2,bcd,[d,d^a]^{\sigma^n},
   [d,d^{acaca}]^{\sigma^n},(n\in\N_0) \Big\rangle,$$
can be constructed as follows. 

\beginexample
gap> F:=FreeGroup( "a", "b", "c", "d" );
<free group on the generators [ a, b, c, d ]>
gap> AssignGeneratorVariables( F );
#I  Assigned the global variables [ a, b, c, d ]
gap> frels:=[ a^2, b^2, c^2, d^2, b*c*d ];;
gap> endos:=[ GroupHomomorphismByImagesNC( F, F, [ a, b, c, d ], [ c^a, d, b, c ]) ];;
gap> irels:=[ Comm( d, d^a ), Comm( d, d^(a*c*a*c*a) ) ];;
gap> G:=LPresentedGroup( F, frels, endos, irels );
<L-presented group on the generators [ a, b, c, d ]>
\endexample

There are various examples of finitely $L$-presented groups available
in the library of the \NQL-package.

\> ExamplesOfLPresentations( <n> ) F

returns some well-known examples of finitely $L$-presented groups. The
input of this function needs to be a positive integer at most $10$.

\beginitems
$n=1$ The Grigorchuk group on 4 generators; cf.~\cite{Grigorchuk80}, 
      \cite{Lysenok85}, and \cite{Bartholdi03}, Theorem 4.6,

$n=2$ the Grigorchuk group on 3 generators; cf.~\cite{Grigorchuk80},
      \cite{Lysenok85}, and \cite{Bartholdi03}, Theorem 4.6,

$n=3$ the lamplighter group $\Z_2\wr\Z$; cf.~\cite{Bartholdi03}, Theorem 4.1,

$n=4$ the Brunner-Sidki-Vieira group; cf.~\cite{BrunnerVieiraSidki99} and
      \cite{Bartholdi03}, Theorem 4.4,

$n=5$ the Grigorchuk supergroup; cf.~\cite{BartholdiGrigorchuk02} and
      \cite{Bartholdi03}, Theorem 4.6,

$n=6$ the Fabrykowski-Gupta group; cf.~\cite{FabrykowskiGupta85} and 
      \cite{BEH08},

$n=7$ the Gupta-Sidki group; cf.~\cite{Sidki87} and 
      \cite{BEH08},

$n=8$ an index-$3$ subgroup of the Gupta-Sidki group

$n=9$ the Basilica group; cf.~\cite{GrigorchukZuk02} and
      \cite{BartholdiVirag05},
 
$n=10$ Baumslag's finitely generated, infinitely related group 
       with a trivial multiplier; cf.~\cite{Baumslag71}.

\enditems

Furthermore, every free group in a variety of groups satisfying finitely
many identities is finitely $L$-presented. Some of these groups are
available from the \NQL-package using the following operations; for
further details we refer to the diploma thesis~\cite{H08}.

\> FreeEngelGroup( <n>, <num> ) O

returns an $L$-presentation for the free $n$-Engel group on <num>
generators; that is, the free group in the variety of <num>-generated
groups satisfying the $n$-Engel identity.

\> FreeBurnsideGroup( <exp>, <num>) O

returns an $L$-presentation for the free Burnside group on <num>
generators with exponent <exp>; that is, the free group in the variety
of <num>-generated groups with exponent <exp>.

\> FreeNilpotentGroup( <c>, <num> ) O

returns an $L$-presentation for the free nilpotent group of class <c> on
<n> generators; that is, the free group in the variety of <num>-generated,
nilpotent groups with nilpotency class $c$.

\> GeneralizedFabrykowskiGuptaLpGroup( <n> ) O

returns an $L$-presentation for the <n>-th generalized Fabrykowski-Gupta 
group as constructed in~\cite{BEH08}.

\> LamplighterGroup( <fil>, <int>) C
\> LamplighterGroup( <fil>, <PcGroup>) C

returns a finite $L$-presentation for the lamplighter group on <int>
lamp states in the first case, if <fil> is the filter `IsLpGroup'. In
the second case, the group <PcGroup> must be a finite cyclic group. Then
the method returns a finite $L$-presentation for the lamplighter group
on `Size(<PcGroup>)' lamp states; for details on the $L$-presentation
see \cite{Bartholdi03}.

\beginexample
gap> LamplighterGroup( IsLpGroup, 2 );
<L-presented group on the generators [ a, t, u ]>
gap> LamplighterGroup( IsLpGroup, CyclicGroup(3) );
<L-presented group on the generators [ a, t, u ]>
\endexample


%%%%%%%%%%%%%%%%%%%%%%%%%%%%%%%%%%%%%%%%%%%%%%%%%%%%%%%%%%%%%%%%%%%%%%%%%%%%
\Section{The underlying free group}

An $L$-presented group is defined as an image of its underlying free
group.  Note that these are two different {\GAP} objects. The elements
of the $L$-presented group are represented by words in the underlying
free group.

\> FreeGroupOfLpGroup( <LpGroup> ) A

returns the underlying free group of the $L$-presented group <LpGroup>. 

\> FreeGeneratorsOfLpGroup( <LpGroup> ) A

returns the generators of the free group which underlies the $L$-presented
group <LpGroup>.

\> GeneratorsOfGroup( <LpGroup> )  O
 
returns the generators of the $L$-presented group <LpGroup>. These are the 
images of the generators of the underlying free group under the natural
homomorphism.

\> UnderlyingElement( <elm> ) O

returns the preimage of an $L$-presented group element <elm> in the
underlying free group. More precisely, each element of an $L$-presented
group is represented by an element in the free group. This method returns
the corresponding element in the free group.

\> ElementOfLpGroup( <fam>, <elm> ) O

returns the element in the $L$-presented group represented by the word
<elm> on the generators of the underlying free group, if <fam> is the
family of $L$-presented group elements.

\beginexample
gap> F:=FreeGroup( 2 );;
gap> G:=LPresentedGroup( F, [ F.1^2 ], [ IdentityMapping( F ) ], [ F.2 ] );;
gap> FreeGroupOfLpGroup( G ) = F;
true
gap> GeneratorsOfGroup( G );
[ f1, f2 ]
gap> FreeGeneratorsOfLpGroup( G );
[ f1, f2 ]
gap> last = last2;
false
gap> UnderlyingElement( G.1 );
f1
gap> last in F;
true
gap> ElementOfLpGroup( ElementsFamily( FamilyObj( G ) ), last2 ) in G;
true
\endexample

%%%%%%%%%%%%%%%%%%%%%%%%%%%%%%%%%%%%%%%%%%%%%%%%%%%%%%%%%%%%%%%%%%%%%%%%%%%%
\Section{Accessing an L-presentation}

The fixed relators, the iterated relators, and the endomorphisms of an
$L$-presented group are accessible with the following methods.

\> FixedRelatorsOfLpGroup( <LpGroup> ) A

returns the fixed relators of the $L$-presented group <LpGroup> as elements
of the underlying free group.

\> IteratedRelatorsOfLpGroup( <LpGroup> ) A

returns the iterated relators of the $L$-presented group <LpGroup>
as elements of the underlying free group.

\> EndomorphismsOfLpGroup( <LpGroup> ) A

returns the endomorphisms of the $L$-presented group <LpGroup> as
endomorphisms of the underlying free group.

\beginexample
gap> F:=FreeGroup( 2 );;
gap> G:=LPresentedGroup( F, [ F.1^2 ], [ IdentityMapping( F ) ], [ F.2 ] );
<L-presented group on the generators [ f1, f2 ]>
gap> FixedRelatorsOfLpGroup( G );
[ f1^2 ]
gap> IteratedRelatorsOfLpGroup( G );
[ f2 ]
gap> EndomorphismsOfLpGroup( G );
[ IdentityMapping( <free group on the generators [ f1, f2 ]> ) ]
\endexample

%%%%%%%%%%%%%%%%%%%%%%%%%%%%%%%%%%%%%%%%%%%%%%%%%%%%%%%%%%%%%%%%%%%%%%%%%%%%
\Section{Attributes and properties of L-presented groups}

For the method-selection of the nilpotent quotient algorithm, an
$L$-presented group may have the following attributes and properties.

\> UnderlyingAscendingLPresentation( <LpGroup> ) A

returns the underlying ascending $L$-presentation of <LpGroup>; that
is, if <LpGroup> is finitely $L$-presented by $(S,Q,\Phi,R)$, the
underlying ascending $L$-presentation is $(S,\emptyset,\Phi,R)$.

\> UnderlyingInvariantLPresentation( <LpGroup> ) A

attempts to compute a 'good' underlying invariant $L$-presentation
for <LpGroup>; that is, if <LpGroup> is finitely $L$-presented by
$(S,Q,\Phi,R)$, then this method seeks to find a subset $Q'\subseteq Q$
such that $(S,Q',\Phi,R)$ is an invariant $L$-presentation. Note
that there is always the underlying ascending $L$-presentation
$(S,\emptyset,\Phi,R)$. However, for the efficiency of the nilpotent
quotient algorithm it is important that the subset $Q'$ is as big as
possible.

Since it is undecidable, in general, whether or not a given $L$-presentation
is invariant, there is no algorithm which can determine the best
possible underlying invariant $L$-presentation. The method implemented for
this attribute tries to compute a 'good' invariant $L$-presentation and
will return the underlying ascending $L$-presentation in the worst case.

This attribute can be set manually using
`SetUnderlyingInvariantLPresentation'. For instance, the Gri\-gorchuk group 
$$ \Big\langle a,b,c,d~\Big|~a^2,b^2,c^2,d^2,bcd,[d,d^a]^{\sigma^n},
   [d,d^{acaca}]^{\sigma^n},(n\in\N_0) \Big\rangle,$$
is invariantly $L$-presented and therefore, it should be constructed
as follows

\beginexample
gap> F:=FreeGroup( "a", "b", "c", "d" );;
gap> AssignGeneratorVariables( F );
#I  Assigned the global variables [ a, b, c, d ]
gap> frels:=[ a^2, b^2, c^2, d^2, b*c*d ];;
gap> endos:=[ GroupHomomorphismByImagesNC( F, F, [ a, b, c, d ], [ c^a, d, b, c ]) ];;
gap> irels:=[ Comm( d, d^a ), Comm( d, d^(a*c*a*c*a) ) ];;
gap> G:=LPresentedGroup( F, frels, endos, irels );
<L-presented group on the generators [ a, b, c, d ]>
gap> SetUnderlyingInvariantLPresentation( G, G );;
\endexample

\> IsAscendingLPresentation( <LpGroup> ) P

checks whether the $L$-presentation of <LpGroup> is ascending; that is,
if the set of fixed relators is empty. This property is set automatically
when creating an $L$-presented group with no fixed relators using the
function `LPresentedGroup'.

\> IsInvariantLPresentation( <LpGroup> ) P

attempts to check whether the $L$-presentation of <LpGroup>
is invariant. In general, one cannot decide whether or not a
given $L$-presentation is invariant. There are mainly two methods
implemented for this property. The first method seeks to find a 'good'
underlying invariant $L$-presentation using the operation
`UnderlyingInvariantLPresentation'. If this latter $L$-presentation
coincides with the $L$-presentation of <LpGroup>, then <LpGroup>
is invariantly $L$-presented. If this method fails, then the second
method uses the nilpotent quotient algorithm for $L$-presented groups
which yields a necessary condition for an $L$-presented group to
be invariantly $L$-presented. Note that the latter method may not
terminate. For instance, both methods fail on Baumslag's finitely
generated, infinitely related group with trivial multiplier returned by
`ExamplesOfLPresentations'.

\> EmbeddingOfAscendingSubgroup( <LpGroup> ) A 

stores an embedding of an ascendingly $L$-presented subgroup of the
$L$-presented group <LpGroup>. This attribute is set for ascending
$L$-presentations only. In this case, the identity map of <LpGroup> is
returned. This attribute is used in the \FR-package which can construct
various finitely $L$-presented groups. The embedding is useful for a
nilpotent quotient algorithm of a non-invariantly $L$-presented group.

%%%%%%%%%%%%%%%%%%%%%%%%%%%%%%%%%%%%%%%%%%%%%%%%%%%%%%%%%%%%%%%%%%%%%%%%%%%%
\Section{Methods for L-presented groups}

\> MappedWord( <x>, <gens>, <imgs> ) O

returns the group element obtained from $x$ by replacing each occurrence
of a generator in <gens> by the corresponding element in the list
<imgs>. The lists <gens> and <imgs> need to have the same length.

\> EpimorphismFromFpGroup( <LpGroup>, <n> ) O

returns an epimorphism from a finitely presented group onto <LpGroup>. The
finitely presented group is obtained from <LpGroup> by applying only
words of length at most <n> in the endomorphisms of <LpGroup> to the
iterated relators of <LpGroup>.

\> SplitExtensionByAutomorphismsLpGroup( <LpGroup>, <H>, <auts>) O

returns an $L$-presentation for the split extension of <LpGroup> by
an $L$-presented or by a finitely presented group <H>. The action of a
generator of <H> on <LpGroup> is given by an automorphism in the list
<auts>. Thus for each generator of <H> there must be an automorphism in
the list <auts>.

\beginexample
gap> F := FreeGroup( "a" );
<free group on the generators [ a ]>
gap> H := F / [ F.1^3 ];
<fp group on the generators [ a ]>
gap> U := ExamplesOfLPresentations( 8 );
<L-presented group on the generators [ t, u, v ]>
gap> aut:=GroupHomomorphismByImagesNC( U, U, [ U.1, U.2, U.3 ], [ U.2, U.3, U.1 ] );
[ t, u, v ] -> [ u, v, t ]
gap> SplitExtensionByAutomorphismsLpGroup( U, H, [ aut ] );
<L-presented group on the generators [ t, u, v, a ]>
\endexample

\> = ( <elm1>, <elm2> ) O

uses the operation `NqEpimorphismNilpotentQuotient' to compare the images
of <elm1> and <elm2> in a nilpotent quotient of the group. The implemented
method successively increases the class of the considered quotient until
the images differ. Hence, this method may not terminate and it will only
determine whether the elements <elm1> and <elm2> are different.

\> AsLpGroup( <G> ) O

returns an ascending $L$-presentation for a finitely presented group <G>
or for a free group <G>.

\> IsomorphismLpGroup( <G> ) O

returns an isomorphism from a finitely presented group <G> or from a free
group <G> to the $L$-presented group obtained from the method `AsLpGroup'.

\> Display( <G> ) O 

prints the $L$-presentation of <G>.

\beginexample
gap> F:=FreeGroup( 2 );
<free group on the generators [ f1, f2 ]>
gap> G:=F/[ F.1^2, F.2^2, Comm( F.1, F.2 ) ];
<fp group on the generators [ f1, f2 ]>
gap> IsomorphismLpGroup( G );
[ f1, f2 ] -> [ f1, f2 ]
gap> Range(last);
<L-presented group on the generators [ f1, f2 ]>
gap> Display(last);
generators = [ f1, f2 ]
fixed relators = [ ]
endomorphism = [
IdentityMapping( <free group on the generators [ f1, f2 ]> ) ]
iterated relators = [
f1^2,
f2^2,
f1^-1*f2^-1*f1*f2 ]
\endexample


%%%%%%%%%%%%%%%%%%%%%%%%%%%%%%%%%%%%%%%%%%%%%%%%%%%%%%%%%%%%%%%%%%%%%%%%%%%%
%%
%E  lpres.tex  . . . . . . . . . . . . . . . . . . . . . . . . ends here
