%%%%%%%%%%%%%%%%%%%%%%%%%%%%%%%%%%%%%%%%%%%%%%%%%%%%%%%%%%%%%%%%%%%%%%%%%%%%
%%
%W  intro.tex			NQL				René Hartung
%%
%H  $Id: preface.tex,v 1.12 2010/09/04 12:54:32 gap Exp $
%%

%%%%%%%%%%%%%%%%%%%%%%%%%%%%%%%%%%%%%%%%%%%%%%%%%%%%%%%%%%%%%%%%%%%%%%%%%%%%
\Chapter{Preface}

In 1980, Grigorchuk~\cite{Grigorchuk80} gave an example of an infinite,
finitely generated torsion group which provided a first explicit
counter-example to the General Burnside Problem. This counter-example
is nowadays called the <Grigorchuk group> and was originally defined
as a group of transformations of the unit interval which preserve the
Lebesgue measure. Beside being a counter-example to the General Burnside
Problem, the Grigorchuk group was a first example of a group with an
intermediate growth function (see \cite{Grigorchuk83}) and was used in
the construction of a finitely presented amenable group which is not
elementary amenable (see~\cite{Grigorchuk98}).

The Grigorchuk group is not finitely presentable
(see~\cite{Grigorchuk99}). However, in 1985, Igor Lysenok
(see~\cite{Lysenok85}) determined the following recursive presentation
for the Grigorchuk group:
$$ \langle a,b,c,d\mid a^2,b^2,c^2,d^2,bcd,[d,d^a]^{\sigma^n},[d,d^{acaca}]
   ^{\sigma^n}, (n\in\N)\rangle,$$
where $\sigma$ is the homomorphism of the free group over $\{a,b,c,d\}$
which is induced by $a\mapsto c^a, b\mapsto d, c\mapsto b$, and
$d\mapsto c$. Hence, the infinitely many relators of this recursive
presentation can be described in finite terms using powers of the
endomorphism $\sigma$.

In 2003, Bartholdi~\cite{Bartholdi03} introduced the notion of an
<$L$-presentation> for presentations of this type; that is, a group
presentation of the form
$$ G=\left\langle S~\left|~ Q\cup \bigcup_{\varphi\in\Phi^\*}
   R^\varphi\right.\right\rangle,$$
where $\Phi^\*$ denotes the free monoid generated by a set of free group
endomorphisms $\Phi$. He proved that various branch groups are finitely
$L$-presented but not finitely presentable and that every free group
in a variety of groups satisfying finitely many identities is finitely
$L$-presented (e.g. the Free Burnside- and the Free $n$-Engel groups).

The {\NQL}-package defines new {\GAP} objects to work with finitely
$L$-presented groups. The main part of the package is a nilpotent quotient
algorithm for finitely $L$-presented groups; that is, an algorithm which
takes as input a finitely$L$-presented group $G$ and a positive integer
$c$. It computes a polycyclic presentation for the lower central series
quotient $G/\gamma_{c+1}(G)$.  Therefore, a nilpotent quotient algorithm
can be used to determine the abelian invariants of the lower central
series sections $\gamma_c(G)/\gamma_{c+1}(G)$ and the largest nilpotent
quotient of $G$ if it exists.

Our nilpotent quotient algorithm generalizes Nickel's algorithm for
finitely presented groups (see~\cite{Nickel96}) which is implemented in
the {\NQ}-package; see~\cite{nq}. In difference to the {\NQ}-package,
the {\NQL}-package is implemented in \GAP\ only.  

Since finite $L$-presentations generalize finite presentations, our
algorithm also applies to finitely presented groups. It coincides with
Nickel's algorithm in this special case.

Our algorithm can be readily modified to determine the $p$-quotients of
a finitely $L$-presented group. An implementation is planned for future
expansions of the package.

A detailed description of our algorithm can be found in~\cite{BEH08}
or in the diploma thesis~\cite{H08} which is publicly available from
the website \URL{http://www.uni-math.gwdg.de/rhartung/pub/index.html}

Further the \NQL-package includes the algorithms of~\cite{Har09}
and~\cite{EH09} for approximating the Schur multiplier and the outer
automorphism group, respectively, of finitely $L$-presented groups.

%%%%%%%%%%%%%%%%%%%%%%%%%%%%%%%%%%%%%%%%%%%%%%%%%%%%%%%%%%%%%%%%%%%%%%%%%%%%
%%
%E  preface.tex  . . . . . . . . . . . . . . . . . . . . . . . . ends here
