%%%%%%%%%%%%%%%%%%%%%%%%%%%%%%%%%%%%%%%%%%%%%%%%%%%%%%%%%%%%%%%%%
\Chapter{Interface to CARAT}

The {GAP} interface to {\CARAT} consists of two parts, low level 
interface routines to {\CARAT} functions on the one hand, and 
comfortable high level {\GAP} functions on the other hand. 
The high level functions, implemented in terms of the low level 
functions, provide actually methods for functions and operations 
declared in the {GAP} library.

Note that while (almost) all {\CARAT} functions should be accessible
from within {\GAP} by the low level interface routines, high level
interface routines are provided only for a small subset of the
{\CARAT} functions. Priority has been given to routines providing
functionality that has previously not been available in {\GAP}.
Further high level interface routines may be added in the future.

%%%%%%%%%%%%%%%%%%%%%%%%%%%%%%%%%%%%%%%%%%%%%%%%%%%%%%%%%%%%%%%%%
\Section{Action from the left and from the right}

In crystallography, the convention usually is that matrix groups
act from the left on column vectors. This convention is adopted
also in {\CARAT}. The low level interface routines described below
must respect this convention and provide {\CARAT} with data in the 
expected format.

On the other hand, in {\GAP} the convention is that all groups 
act from the right, in the case of matrix groups on row vectors. 
However, in order to make {\GAP} accessible to crystallographers,
functions that are important in crystallography and for which it
matters which action is assumed, are provided in two variants, 
one for each convention. The high level routines currently provided
by this package do not depend on which convention is assumed.
This may change, however, when further high level routines are 
added in the future.

%%%%%%%%%%%%%%%%%%%%%%%%%%%%%%%%%%%%%%%%%%%%%%%%%%%%%%%%%%%%%%%%%
\Section{CARAT input and output files}

{\CARAT} routines read their input from one or several input files,
and write the result to standard output. In order to use {\CARAT}
routines from within {\GAP}, the input must be prepared in suitably
formatted input files. A {\CARAT} command is then executed with these
input files, with standard output redirected to an output file, 
which is read back into {\GAP} afterwards. This section describes
routines interfacing with {\CARAT} input and output files.

Working with {\CARAT} requires many temporary files. When the {\CARAT}
package is loaded, a temporary directory is created, where one can
put such files. The routine

\>CaratTmpFile( <filename> ) F

returns a file name <filename> in the {\CARAT} temporary directory, which
can be used to store temporary data. Of course, it is also possible
to use any other file name, for instance files in the current directory.

\>CaratShowFile( <filename> ) F

displays the contents of any text file on the terminal. This can be
used to inspect the contents of {\CARAT} input and output files.

Most {\CARAT} data files are in either of two formats. The first {\CARAT}
file type is the Matrix File, containing one or several matrices.
The following routines serve as interface to {\CARAT} Matrix Files.

\>CaratWriteMatrixFile( <filename>, <data> ) F

takes a file name and a matrix or a list of matrices, and writes the
matrix or matrices to the file.

\>CaratReadMatrixFile( <filename> ) F

reads a {\CARAT} matrix file, and returns a matrix or a list of matrices
read from the file.

The second {\CARAT} file type is the Bravais File, containing information
on a finite unimodular group. In { \GAP}, the contents of a Bravais File 
is represented by a Bravais record, having the following components:
\beginitems
`generators'   & generators of the finite unimodular group

`formspace'    & basis of the space of invariant forms (optional)

`centerings'   & list of centering matrices (optional)

`normalizer'   & additional generators of the normalizer in GL(n,Z) (optional)

`centralizer'  & additional generators of the centralizer in GL(n,Z) (optional)

`size'         & size of the unimodular group (optional)
\enditems

The following routines serve as interface to {\CARAT} Bravais Files.

\>CaratWriteBravaisFile( <filename>, <data> ) F

takes a file name and a Bravais record, and writes the data in the
Bravais record to the file.

\>CaratReadBravaisFile( <filename> ) F

reads a Bravais File, and returns the resulting Bravais record.

Certain {\CARAT} programs produce output files containing several Bravais 
records, possibly preceeded by a varying number of header lines.

\>CaratReadMultiBravaisFile( <filename> ) F

reads such a multi-Bravais file, and returns a record with the components
`info' and `groups'. `info' is the list of header lines before the first
Bravais record starts, and `groups' is the list of Bravais records read from
the file.

%%%%%%%%%%%%%%%%%%%%%%%%%%%%%%%%%%%%%%%%%%%%%%%%%%%%%%%%%%%%%%%%%%%
\Section{Executing CARAT commands}

To execute a {\CARAT} program from within {\GAP}, some low level,
general purpose routines are provided in this package. 
Higher level routines for certain {\CARAT} functions may be available 
in the {\GAP} library or in other packages. These higher
level functions are expected to use the following low level routines,
so that changes in the low level interface will be transparent. 

An arbitrary {\CARAT} program can be executed with the routine

\>CaratCommand( <command>, <args>, <outfile> ) F

where <command> is the name of a {\CARAT} program, <args> is a string
containing the command line arguments of the {\CARAT} program,
and <outfile> is the name of the file to which the output is to be 
written. Example:

\begintt
gap> CaratCommand( "Z_equiv", "file1 file2", "file.out" );
\endtt

A short description of the arguments and options of any {\CARAT} 
program can be obtained from the {\CARAT} online help facility with

\>CaratHelp( <command> ) F

where <command> is the name of the {\CARAT} program. CaratHelp executes
the program with the `-h' option, and writes the output to the 
terminal. Example:

\begintt
gap> CaratHelp( "Z_equiv" );
\endtt

A list of all {\CARAT} programs, along with a description of their
usage and functionality, can be found in the {\CARAT} documentation 
(in HTML), in the file

\begintt
documentation/introduction.html
\endtt

in the {\CARAT} home directory.


%%%%%%%%%%%%%%%%%%%%%%%%%%%%%%%%%%%%%%%%%%%%%%%%%%%%%%%%%%%%%%%%%%%
\Section{Methods provided by CARAT}

{\CARAT} implements methods for the following functions and operations
declared in the {\GAP} library. For a detailed description of these
functions, please consult the {\GAP} manual (section 
"ref:Matrix Groups in Characteristic 0").

\>BravaisGroup( <G> ) A

\>IsBravaisGroup( <G> ) P

\>BravaisSubgroups( <G> ) A

\>BravaisSupergroups( <G> ) A

\>NormalizerInGLnZBravaisGroup( <G> ) A

\>`Normalizer( GL(<n>, Integers), <G> )'{Normalizer!in GLnZ}@{in GLnZ} O

\>`Centralizer( GL(<n>, Integers), <G> )'{Centralizer!in GLnZ}@{in GLnZ} O

\>ZClassRepsQClass( <G> ) A

\>`RepresentativeAction( GL(<n>,Integers), <G1>, <G2> )'{RepresentativeAction!in GLnZ}@{in GLnZ} O


%%%%%%%%%%%%%%%%%%%%%%%%%%%%%%%%%%%%%%%%%%%%%%%%%%%%%%%%%%%%%%%%%%%
\Section{CARAT Bravais Catalog}

{\CARAT} contains a catalog with $\Z$-class representatives of all
Bravais groups of dimension up to 6. These Bravais groups are
accessed via a crystal family symbol.

\>`CaratCrystalFamilies[d]'{CaratCrystalFamilies}@{`CaratCrystalFamilies'} V

returns a list of inequivalent crystal family symbols in dimension <d>.

\>BravaisGroupsCrystalFamily( <symb> ) F

returns a list of $\Z$-class representatives of the Bravais groups
in the crystal family with symbol <symb>.

%%%%%%%%%%%%%%%%%%%%%%%%%%%%%%%%%%%%%%%%%%%%%%%%%%%%%%%%%%%%%%%%%%%
\Section{CARAT Q-Class Catalog}

{\CARAT} contains a catalog with representatives of all $Q$-classes of
finite unimodular groups up to dimension~6. This catalog can be accessd
with the function

\>CaratQClassCatalog( <grp>, <mode> ) F

where <grp> is a finite unimodular group of dimension up to 6, and
<mode> is an integer. This function returns a record with one or
several of the following components, depending on the decomposition of
$mode = n_{0} + n_{1} * 2 + n_{2} * 4$ into powers of 2:
\beginitems
`qclass'     & Q-class symbol; this component is always present

`familysymb' & crystal family symbol (present if $n_{0} \<> 0$)

`trans'      & transformation to standard representative of Q-class: 
               <grp>\^{}<trans> = <std>
               (present if $n_{1} \<> 0$)

`group'      & standard representative of Q-class of <grp> 
               (present if $n_{2} \<> 0$ )
\enditems

If <G1> and <G2> are two unimodular groups,

\>ConjugatorQClass( <G1>, <G2> ) F

returns a rational matrix <m> such that <G1>\^{}<m> = <G2>, or fail, if 
the groups are not in the same Q-class. Since this function uses the 
{\CARAT} Q-class catalog, only groups up to dimension 6 are supported. 
If this dimension is exceeded, an error is reported.


