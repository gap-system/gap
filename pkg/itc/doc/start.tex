%%%%%%%%%%%%%%%%%%%%%%%%%%%%%%%%%%%%%%%%%%%%%%%%%%%%%%%%%%%%%%%%%%%%%%%%%
%%
%W  start.tex              GAP documentation            Joachim Neub"user
%%
%H  $Id: start.tex,v 1.8 2003/06/11 16:41:06 gap Exp $
%%
%Y  Copyright (C) 1999, Lehrstuhl D fuer Mathematik, RWTH Aachen, Germany
%%


%%%%%%%%%%%%%%%%%%%%%%%%%%%%%%%%%%%%%%%%%%%%%%%%%%%%%%%%%%%%%%%%%%%%%%%%%
\Chapter{How to Start?}

%%%%%%%%%%%%%%%%%%%%%%%%%%%%%%%%%%%%%%%%%%%%%%%%%%%%%%%%%%%%%%%%%%%%%%%%%
\Section{Starting XGAP}

You begin an {\ITC} session by starting {\XGAP}. On most systems this
is done by completing the provided line

\begintt
user@host:~>
\endtt

to

\begintt
user@host:~> xgap
\endtt

on the command line. Ask your system administrator if this does not
work. This command will create a new window, the so called {\GAP}
Window, in which {\GAP} is awaiting your input.

During a typical {\ITC} session various windows will be opened to
display the involved tables and further information. The size of the
characters displayed in these windows and hence the sizes of the
windows themselves are determined by the so-called *fonts* used.

The {\XGAP} package knows five different fonts, `tiny', `small',
`normal', `large', and `huge' (see the {\XGAP} manual for details), but
the {\ITC} routines use only two of them, namely `small' and `normal'.
So, if the default {\ITC} window sizes are inconvenient for you, you
need only to change the values of these two fonts. You may do this by
starting {\XGAP} with one or two additional arguments of the form
\"`-small <font>'\" or \"`-normal <font>'\" where <font> has to be a
font which is known to your system. You can get a list of all available
fonts via the command

\begintt
user@host:~> xlsfonts
\endtt

Example:

\begintt
user@host:~> xgap -small 6x12 -normal 8x13
\endtt

In particular, you should use this option to change the fonts if the
default fonts of {\XGAP} do not work properly in your local installation
(this may happen in some rare cases where there are too few or too many
fonts).

%%%%%%%%%%%%%%%%%%%%%%%%%%%%%%%%%%%%%%%%%%%%%%%%%%%%%%%%%%%%%%%%%%%%%%%%%
\Section{Starting ITC}

At the top of the {\GAP} Window opened by {\XGAP} you will see a
``title bar'' with three buttons: `GAP', `Run', and `Help'. Clicking
either of these will pop up a menu, in which you may choose between
several options.

You can always interrupt a computation by making the {\GAP} Window
active and pressing <CTRL-C>. You will get the same effect selecting
`Interrupt' in the `Run' menu.

In the {\GAP} Window, as usual, the {\GAP} prompt

\begintt
gap>
\endtt

will indicate that {\GAP} is awaiting your input.

If it is not loaded automatically, first load the {\ITC} package by

\begintt
gap> RequirePackage( "itc" );
\endtt

See~"ref:RequirePackage" in the {\GAP} Reference Manual.

Then you may input a finitely presented group, say <g>, and a
subgroup, say <h>, of it, the cosets of which you want to enumerate
using {\ITC}. The format of this input is the same as usual in {\GAP},
see Chapter "ref:Finitely Presented Groups" of the {\GAP} reference
manual for the details.

{\ITC} will in several of its windows have to display generators of
the free group which is used to describe the presentation. Note that
{\ITC} will use the *names* of the generators of this free group, if
these names consist of at most three symbols. Otherwise {\ITC} will
use names of the form $g1$, $g2$, ... up to at most $g99$ for the
generators of the free group (if the group has more than 99
generators, an error message will be given). A choice of short and
easily distinguishable names for the generators of the free group is
hence advisable.

Now you are ready to start the {\ITC} using the command `InteractiveTC'.

\>InteractiveTC( <g>, <h> ) F

Called with a group <g> and a subgroup <h>,
this will open the ``Coset Table Window'', which really serves a
multiple purpose:

\beginlist

\item{--}
  It shows the Coset Table and allows certain operations by mouse click
  in this Coset Table. We will describe the details in Section "The
  Coset Table" of Chapter "The Tables". In this chapter also a
  description of all other tables is given that can be shown on the
  screen by {\ITC}.

\item{--}
  Directly below the Coset Table space is left for one row of text in
  which some warnings will be given (in red) by {\ITC} if the user tries
  to invoke actions that in the present state are not possible, e.~g. if
  the user tries to call the short-cut function while the tables have
  not yet closed (see Section~"Warnings").

\item{--}
  It provides up to date information on the state that an interactive
  coset enumeration has reached in the ``Information Line'' (see~"The
  Information Line").

\item{--}
  It also provides a number of buttons which can be used in the
  interactive handling of a coset enumeration by {\ITC}. We will
  describe this aspect in a separate chapter ``Buttons and Menus''
  (Chapter~"Buttons and Menus").

\endlist

The initial table size for the {\ITC} Coset Table is set by default to
1000. You can change this default table size using the menu entry
`change default table size' in the menu of the top button `Settings',
(see~"change default table size"). If during a run of a CE the table
size does not suffice, you can extend it using the menu entry `extend
table size' (see~"extend table size") in the same menu.


%%%%%%%%%%%%%%%%%%%%%%%%%%%%%%%%%%%%%%%%%%%%%%%%%%%%%%%%%%%%%%%%%%%%%%%%%
%%
%E

