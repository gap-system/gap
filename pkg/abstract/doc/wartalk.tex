\documentclass[12pt]{article}
\usepackage{lslide}
\usepackage{a4wide}
\usepackage{times}
\usepackage{helvetic}
\usepackage{epsf}
\landscape
\vertgroup
\parskip 1ex plus 1ex minus 0.2ex
\def\bs{\begin{slide}}
\def\es{\end{slide}}
\def\bi{\begin{itemize}}
\def\ei{\end{itemize}}
\def\GAP{\textsf{GAP}}
\begin{document}
\sf
\title[Knowledge-based Approach]{A Knowledge Based Approach to Algebraic Computation}
\author{Steve Linton}
\organization{Division of Computer Science, St.~Andrews}
\date{June 2000}
\titlepage
\bs
\subsection{Algorithms}

An algorithm is usually presented as:
\bi
\item Input specification 
\bi
\item As simple as possible eg a set $A$ of $s$ permutations of degree $n$ and one more permutation $x$
\ei
\item Output specification -- 
\bi
\item Again simple eg true iff $x$ lies in $\langle A\rangle$
\ei
\item Performance guarantees or expectations
\ei

Although this format is a good match for software engineering and
publishing papers, it may not be such a good match with applications
in mathematical problem-solving.
\es
\bs
\subsection{Knowledge and Problem Solving}
\bi

\item In many situations, a mathematician has a lot of knowledge about
the structure of interest, and the problem is to fill in a small gap
in it, or to make concrete some abstractly known structure

\bi
\item Knowing that some subgroup
of a well-known group has structure $2S_5$, which of the two isoclinic
groups of this shape is it?

\item Find generators for the centraliser of an
element, given that we know the conjugacy class of the element and the
abstract structure of the centraliser
\ei
\item Also, any useful results we compute along the way should not be discarded
\ei
\es
\bs
\subsection{The Art and Science}

\bi
\item In 1992, ``The Art and Science of Computing in Large Groups''
discussed approaches and techniques for solving this kind of problem
by hand in groups too large for general-purpose automated methods to
be effective.

\item I am (funded by the Royal Society of
Edinburgh -- thanks!)  looking at the issues involved in automating
these ideas


\item Thomas Breuer is looking at related issues in the context of
linking Rob Wilson's ``ATLAS of Group Representations'' to \GAP.
\ei
\es

\bs
\subsection{The Art and Science -- Principles}
\begin{enumerate}
\item Know your group
\item Use what you know
\item Do no more than you have to
\item Work in subgroups whenever possible
\end{enumerate}
\es
\bs
\subsection{Issues in Automating the Art and Science}
\begin{enumerate}
\item Knowledge representation
\item Computational building blocks
\item Algorithm synthesis
\item Recognition and constructive recognition
\end{enumerate}
\es
\bs
\subsection{Knowledge Representation}
\bi
\item Knowledge is intended to be central in this approach
\item Both ``old'' knowledge, found from databases, etc. and ``new''
knowledge, computed by the user
\item {\GAP} 4 already makes more use of new knowledge than previous
systems
\item Many new algorithms depend on CFSG in the explicit form of 
``old knowledge'' --  databases of element order distributions, 
presentations or recognition algorithms.
\ei
\es
\bs
\subsection{Old Knowledge}
\bi
\item A great deal is known about a great many groups (and also Lie
algebras, Hecke algebras, etc.)
\item Quite a lot of it is machine-readable in one or another way
\item There are a lot of problems in putting it together,
\bi
\item Variety of data formats and storage conventions
\item Variety of choices of generating set, basis or representation
\item Subtle inter-dependencies -- for example generality issues in
modular character tables
\item Actual errors and their propagation
\ei
\item Recent work is improving the situation considerably -- The ATLAS 
of Group Representations; The \GAP\ databases
\ei
\es
\bs
\subsection{New Knowledge}
\bi
\item Even {\GAP} does not make as much use as it could of new
knowledge
\item Even if I recognize two groups as the same simple group, results 
of computation about one will not pass to the other.
\item The problem is that {\GAP} deals almost entirely with
``concrete'' groups, while much knowledge is applicable to
``abstract'' groups (isomorphism classes).
\ei
\es
\bs
\subsection{A Framework for Knowledge}
\bi
\item When we have a matrix or permutation ``group'' in a program, we really have five
things
\begin{description}
\item[Abstract Group --] An isomorphism class of groups
\item[Generated Group --] A group with an explicit generating set,
defined up to isomorphism of such
\item[Abstract Representation -- ] A matrix or permutation representation
of the abstract group, defined up to equivalence 
\item[Concrete Representation -- ] A matrix or permutation representation
with a fixed basis (or point ordering)
\item[Concrete Group -- ] The combination of a Concrete Representation and 
a Generated Group
\end{description}
\ei
\es
\bs
\subsection{A Framework for Knowledge 2}
\bi
\item Examples of Information that are naturally stored with each:
\begin{description}
\item[Abstract Group:] Size; name; abstract composition factors;
standard generator specification; character table
\item[Generated Group:] Words [SLP] to reach standard generators; words for
generators of subgroups
\item[Abstract Representation:] character; abstract decomposition
\item[Concrete Representation:] orbit representatives; block systems;
irreducible subspaces; quadratic forms
\item[Concrete Group:] explicit generators (matrices or permutations)
not much more
\end{description}
\ei
\es
\bs
\subsection{Knowledge Representation -- Picture}
\begin{center}
\epsfbox{Knowledge.eps}
\end{center}
\es
\bs
\subsection{Physical Knowledge Representation}
\bi
\item the ``old'' data behind this abstract picture needs to be stored somewhere, and accessed on demand. 
\item The data is mixed:
\bi
\item Detailed and bulky data about a few large groups
\item Parametrised data about families of groups
\item Compact data about many small groups, where almost everything can be recomputed quickly on demand
\ei
\item Need a uniform interface layer, on top of a variety of storage methods
\item Also want storage to be future-proof and portable
\ei
\es
\bs
\subsection{Other Knowledge Issues 1}
\bi
\item Origin and dependency tracking 
\bi
\item Mistakes do happen, and we need an audit trail for our results 
\item Needs to be intelligent -- when we can check something directly,
we can prune off much of the audit trail
\item this might also be useful technology 
for managing uncertainty in Monte Carlo/ Las Vegas computations.
\ei
\item Publishing and Sharing
\bi
\item Artificial distinction between ``new'' and ``old'' knowledge -- if you've found something, why not add it to the database
\item Internet access to large databases or to get latest versions
\ei
\ei
\es
\bs
\subsection{Other Knowledge Issues 2}
\bi
\item Partial Information
\bi
\item The ATLAS has the concept of the ``shape'' of a group -- an expression like $3^{1+4}.[2^7.3]$ or $2\cdot 2^{4+6} : S_5$
\item This typically gives a composition series, but only partial information about actions, splitting, isoclinism, etc.
\item This kind of information is useful, and we want to represent and use it
\ei
\ei
\es
\bs
\subsection{Automating Computations}
\bi
\item Writing down algorithms to solve high-level problems efficiently 
in every possible state of knowledge would be impossible
\item instead we resort to an idea from the theorem proving and AI
communities -- planning
\item We build up collection of ``mini-algorithms'' each with inputs,
outputs, requirements and time and space consumption 
\item The planner combines them to solve the users problem
\item The planner might be guided by a library of ``tactics''
\item As a first step, the user might produce a plan and the system
just check it and estimate the resources required. 
\ei
\es
\bs
\subsection{Some Possible Mini-algorithms 1}
\bi
\item Finding an element in a conjugacy class by random search
\item Finding an element in a conjugacy class by powering
\item Finding an element in a conjugacy class by conjugating one you already have
\item Techniques for showing tht an element is not in a conjugacy  class
\item Techniques for showing that a relation between elements does NOT hold
\item Finding an element in the centraliser of an involution using the dihedral groups trick
\ei
\es
\bs
\subsection{Some Possible Mini-algorithms 2}
\bi
\item Finding an element conjugating one involution to another
\item Finding a non-cyclic proper subgroup containing a given element
\item Various ways of showing that two elements do not generate a group isomorphic to a given abstract group
\item Generating elements of a point stabilizer from coincidences in the orbit algorithm
\ei
\es
\bs
\subsection{Some Tactics for Finding Subgroups}
\bi
\item If the subgroups is cyclic, see finding elements in conjugacy classes
\item If the subgroup is not maximal, find a maximal subgroup containing it

\item If you can find a plan for generating elements in the subgroup, then do 
so, until you can prove that they cannot all lie in a smaller subgroup

\item First find its preimage in some proper quotient 
(eg an action on an orbit) then look in there for the actual group.
\ei
\es
\bs
\subsection{Computation -- Other Issues}
\bi
\item When to Give Up Being Clever?
\bi
\item We refine hard questions about big groups into easier ones or ones about smaller groups
\item At some point the overhead of being clever and using the knowledge-base exceeds the gains from doing so
\item At that point it is better to just hand the problem to general-purpose algorithms
\ei
\ei
\es
\bs
\subsection{Passing from Concrete to Abstract}
\bi
\item We want to apply these techniques not only to groups originating from the databases
\item Our techniques will also give us ``known'' subgroups with random generating sets 
\item This means that we need to take concrete groups however arising and attach them to our abstract model
\item This is the problem of group recognition, which has had much recent attention
\ei
\es
\bs
\subsection{Recognition}
\bi
\item Non-constructive recognition simply establishes the correct Abstract Group
\item Further steps fill in the other parts of the Knowledge model and establish more detailed links between concrete and abstract
\item For example, it is often quite easy to link concrete and abstract conjugacy classses by cycle or eigenspace structure

\ei

\es
\bs
\subsection{Recognition Continued}
\bi
\item Ultimate goal is  constructive recognition of both the group and the representation
\item Ideally this wanted not just for simple groups or irreducible representations
\item But, many partial results can still be useful -- eg conjugacy classes above
\item We need to codify, represent and use partial information nicely.
\ei
\es
\bs
\subsection{Summary}
\bi
\item Attempting to automate the approach of ``Art and Science''
\item In this scheme a knowledge base lies at the heart of the system
\item Interesting issues in designing and implementing the knowledge-base -- could be useful in its own right
\item Synthesise ``bespoke'' algorithms from a toolkit of simple components
\item Need to codify these components -- requirements, outputs, resources
\item Recognition algorithms of various degrees of constructivity become very important
\ei

\es
\bs
\subsection{Conclusions}

\bi
\item Much too early to have conclusions.
\ei
\es

\end{document}












