\documentclass{article}
\usepackage{times}
\title{Abhstract objects and Knowledge in GAP -- Design study}
\author{Steve Linton}
\def\GAP{\textsf{GAP}}
\parskip \medskipamount
\parindent 0pt
\begin{document}
\maketitle

\section{Overview}

A design study for a \GAP\ representation of abstract groups,
characters, generated groups, and concrete groups, layered above the
actual \GAP\ group objects. The goal is to implement the ideas in the
Warwick talk.

\section{Principal New Kinds of Objects and  their main attributes}

\subsection{The five layers}

\begin{description}
\item{Abstract groups} name, size, shape, favourite representation, std
generators, is simple, is abelian, is nilpotent, etc char table, table 
of marks

\item{Character} abs group, type (perm or mx) degree, character, symbol, 
is irreducible, composition series, kernel (as abstract subgroup), etc. std basis

\item{Representation} character, transformation to from std basis?,
stabilized forms, etc.

\item{Generated group} words to/from std generators, words for subgroups,
words for conj class reps, presentations, .....

\item{Concrete group} actual GAP group, generated group, representation

\end{description}

\subsection{Other things}
\begin{description}
\item{Shapes} this is a tough one, follow the ATLAS constructions, I
suppose
\item{Shapes of Representations} that is abstract decompositions
\item{Mappings} between generated groups -- by lists of words
\item{Mappings} between abstract groups?? -- recording conjugacy class
embeddings, perhaps?
\item{Elements} fully abstract, or in generated groups or concrete
\item{Conjugacy classes}
\end{description}

\section{Algorithms}

\begin{itemize}
\item Populate the abstract data from a concrete permutation or matrix 
group, propagate it into multiple representations, etc.
\item Populate it from the Atlas rep databases and the char table and
table of marks and likewise propagate
\item Work with shapes and abstract subgroups/mappings
\item Associate (subgroups of) concrete groups with abstract ones
using std gens etc.
\item Merge data between abstract groups on discovering a concrete
isomorphism
\item compute in concrete groups, tracking SLPs (or use chains)
\end{itemize}

\section{Existing Technology}

There is highly relevant technology in the character table library,
the tables of marks library and the AtlasRep share
package. Eventually, one would want to integrate smoothly with all
this, but, time being pressing, for a prototype, we will layer over
all of thise, using the functionality associated with these libraries
where it is helpful.






\end{document}