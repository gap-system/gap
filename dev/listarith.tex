\documentclass{article}
\usepackage{times}
\usepackage{a4wide}
\usepackage{amsfonts}
\parindent 0pt
\parskip0.5\medskipamount
\title{Arithmetic of Lists and Matrices\\
A Description of what the Kernel does and Notes on Implementing the
New Functionality}
\author{Steve Linton}
\begin{document}
\maketitle
\section{Basics -- jump tables}

When the interpreter evaluates an arithmetic expression it reaches a
range of functions such as EvalAInv (\verb|exprs.c|), and from there
(and elsewhere in the kernel) the macros such as AINV and SUM
(\verb|ariths.h|) are called. These dispatch through the kernel jump
tables such as \verb|SumFuncs|. Exceptionally, \verb|EvalSum|, \verb|EvalDiff| and
\verb|EvalProd| fast-track the case of two small integer arguments and
small integer result. 

The jump tables  are indexed by the TNUMs of their
one or two arguments. Certain entries, relating entirely to kernel
objects (types in the range \verb|FIRST_REAL_TNUM| to
\verb|LAST_REAL_TNUM| are installed there, permanently. 
Certain additional entries are initialized via a call to function like
\verb|InstallZeroObject|. This takes one argument specifying whether
to install the verbose or silent versions of the functions. These
\verb|InstallXXXObject| functions also also called by
\verb|ChangeDoOperations| when tracing is enabled or disabled. They
set the table entries for the cases when one or more argument  is of
type between \verb|FIRST_EXTERNAL_TNUM| and \verb|LAST_EXTERNAL_TNUM|
or of types \verb|T_OBJECT|, \verb|T_PREC| or \verb|T_PREC+IMMUTABLE|.


The values installed by default are usually the kernel functions
\verb|XXXObject| which simply call |DoOperation| for the appropriate
operations. There are a few exceptions, for purely internal objects:
\verb|DiffDefault|, \verb|QuoDefault|, \verb|LQuoDefault| and
\verb|CommDefault|, \verb|PowDefault|  are installed, reducing these
operations (\verb|POW| in its conjugate sense) to simpler
ones. \verb|QuoDefault| and \verb|LQuoDefault|, but not the others
include a mutability adjustment on the intermediate value.

\section{TNUM's of Lists}

Kernel lists have TNUM's between \verb|FIRST_LIST_TNUM| and
\verb|LAST_LIST_TNUM| within that range \verb|FIRST_PLIST_TNUM| up to
\verb|LAST_PLIST_TNUM| are plain lists, about which various things may
or may not be known. The rest includes, ranges, blists and strings,
which don't trouble us here.  Important to note is that many plist
tnum's are ``incomplete'' meaning that there is unknown (or
unstorable) information about the list which could change its
behaviour. Note that not knowing sortedness does not make a list
incomplete, as sortedness never affects the results of operations.

Non-kernel objects which are lists will have tnum \verb|T_COMOBJ|,
\verb|T_POSOBJ| or \verb|T_DATOBJ| and a type containing
\verb|IsList|. While they are almost always dealt with in the library,
plain lists containing them may need to be thought about in the
kernel.

After these come the ``virtual'' TNUMS \verb|T_OBJECT|, \verb|T_MAT_CYC| and
\verb|T_MAT_FFE|. I think these are used to report non-stored or
non-storable information about lists during arithmetic method
selection, but we will come to this.

\section{Special Jump Table Entries for Lists}

\subsection{In listoper.c}

\verb|listoper.c| installs some entries in the jump tables,
overwriting what was there before. This is controlled by a vector
\verb|CAT| which classifies all TNUMs (including the virtual ones)
into: scalar, vector, matrix, empty, ``nothing'' (non-arithmetic types)
and incomplete.

\textbf{Note:} Some types are labelled incomplete which probably
should not be: \verb|T_PLIST_NDENSE| and \verb|T_PLIST_DENSE_NHOM_*|. These should
probably be ``nothing'' (ie install no kernel methods). 

For Zero and AInv, the methods installed are then \verb|XxxList| for
incomplete types and XxxListDefault for all other non-scalar types. 
XxxxList, calls \verb|XTNum| to get more information about the list
and then redispatches. XXXListDefault does element-by-element zeroing
or negation, calling ZERO or AINV on each list entry. 

For One, the functions are OneList for incomplete types and OneMatrix
for matrices. Inv is similar. PowMatrixInt is installed for powering
of matrices by integers of any sign. 

For Sum, there are four functions, SumList, which redispatches and
SumListScl, SumSclList and SumListList which implement the three
possible addition algorithms. SumList is installed if either argument
is incomplete, else SumListScl or SumListScl if one argument is a
scalar and the other not, else SumListList if both are vectors or
empty or matrices. Diff is like Sum.

For Prod, there are again four functions, ProdList redespatches,
ProdListScl and ProdSclList do the obvious and ProdListList is inner
product. ProdList is installed if either argument is incomplete,
ProdListScl or ProdSclList if just one argument is scalar,
ProdListList for vector* anything else, ProdListScl
for matrix * vector and ProdSclList for matrix * empty. 

Quo just has QuoList installed if eithert argument is incomplete,
likewise Lquo and Pow. Additionally ProdListList is installed for
\verb|vector ^ matrix|, and PowDefault (conjugate) for matrix * matrix. Comm
is like Quo.

\subsection{In vector.c and vecffe.c}

In \verb|vector.c|, special methods are installed for arithmetic with
arguments of TNum \verb|T_PLIST_CYC_*| and/or \verb|T_INT| for Sum, Diff and
Prod. Also \verb|ProductVectorMatrix| is installed for vectors and
\verb|T_MAT_CYC|.

In \verb|vecffe.c| something very similar is done for
\verb|T_PLIST_FFE| and relatives and the corresponding matrix type.



\section{XTNum extended dispatch}

XTNum is defined in lists.c. Its behaviour is controlled by entries
in the IsXTNumListFuncs jump table. These are boolean (C style)
unctions, possibly defined for types from \verb|FIRST_REAL_TNUM| up to
\verb|LAST_REAL_TNUM|. Each one is tried in reverse TNUM order, with
\verb|T_OBJECT| as the fallback.

In fact, only a few functions are installed, in vector.c and
vecffe.c. In the order in which they are tried, this is for \verb|T_MAT_FFE|,
\verb|IsXTNumMatFFE|, for \verb|T_MAT_CYC|, \verb|IsXTNumMatCyc|, for \verb|T_PLIST_FFE|
\verb|IsXTNumPlistFFE| and for \verb|T_PLIST_CYC|, \verb|IsXTNumPlistCyc| and finally
\verb|IsXTNumEmpty| for \verb|T_PLIST_EMPTY|. The two for vector types also change
the TNUM of the list appropriately. The two for matrix types call the
corresponding vector testers on each row.

\section{Determining the Type of Plists}
If either before or after extended dispatching, we fall into one of
the cases where no kernel method is installed, we usually come to one
of the handlers like SumObjects, which calls out to the SUM operation,
in the library. Note that some things, like replacing a - b by a +
AInv(b), may happen first as noted above.

The operation dispatcher will then immediately determine the types of
the arguments, which, for internal objects takes us to the kernel
jump table \verb|TypeObjFuncs|. For plain lists, the relevant entries
in this are installed in \verb|plist.c|. There are a lot of
these. Rather than rewrite it, I copy here the  explanation from
\verb|plist.c|.
\begin{verbatim}
/****************************************************************************
**
*F  TypePlist(<list>) . . . . . . . . . . . . . . . . .  kind of a plain list
**
**  'TypePlist' returns the kind of the plain list <list>.
**
**  'TypePlist' is the function in 'TypeObjFuncs' for plain lists.
**
**  TypePlist works with KTnumPlist to determine the type of a plain list
**  Considerable care is needed to deal with self-referential lists. This is
**  basically achieved with the OBJ_FLAG_TESTING flag in the Tnum. This must be set in
**  the "current" list before triggering determination of the Type (or KTnum)
**  of any sublist.
**
**  KTnumPlist determined the "true" Tnum of the list, taking account of such
**  factors as denseness, homogeneity and so on. It modifies the stored Tnum
**  of the list to the most informative "safe" value, allowing for the
**  mutability of the list entries (and preserving OBJ_FLAG_TESTING).
**
**  Here begins a new attempt by Steve to describe how it all works:
**
**  We begin with the TNUMs attached to the objects. They are defined in
**  objects.h and consist of the following, each of which can be qualified by
**  adding the constant IMMUTABLE.
**
**   T_PLIST                    nothing is known
**   T_PLIST_NDENSE             known to have a hole
**   T_PLIST_DENSE              known only not to have a hole
**   T_PLIST_DENSE_NHOM         known to be dense but not homogenous *  **
**   T_PLIST_DENSE_NHOM_SSORT   dense, non-hom but strictly sorted
**   T_PLIST_DENSE_NHOM_NSORT   dense, non-hom, known not be be sorted
**   T_PLIST_EMPTY              the empty list
**   T_PLIST_HOM                known to be homogenous *
**   T_PLIST_HOM_NSORT           etc
**   T_PLIST_HOM_SSORT           etc
**   T_PLIST_TAB                known to be a table  *
**   T_PLIST_TAB_NSORT           etc
**   T_PLIST_TAB_SSORT           etc
**   T_PLIST_TAB_RECT           known to be a rectangular table  *
**   T_PLIST_TAB_RECT_NSORT      etc
**   T_PLIST_TAB_RECT_SSORT      etc
**   T_PLIST_CYC                known to be a list of constant kernel cyclotomics
**   T_PLIST_CYC_NSORT           etc
**   T_PLIST_CYC_SSORT           etc
**   T_PLIST_FFE                known to be a list of kernel FFEs
**
**   * -- these tnums can only be safely given when none of the elements of the list
**        is mutable
**   ** -- dense recursive lists (have themselves as a (possibly nested) subobject)
**         appear here
**
**  There are 10 functions entered in TypeObjFuncs:
**      1. TypePlist
**      2. TypePlistNDenseMut/Imm
**      3. TypePlistDenseMut/Imm
**      4. TypePlistDenseNHomMut/Imm
**      5. TypePlistDenseNHomSSortMut/Imm
**      6. TypePlistDenseNHomNSortMut/Imm
**      7. TypePlistEmptyMut/Imm
**      8. TypePlistHom     -- also handles Tab and RectTab
**      9. TypePlistCyc
**      10.TypePlistFfe
**
**     Of these:
**         3 is actually an alias of 1
**         2,4, 5, 6 and 7  simply return a fixed type
**         Thus 1, 8, 9 and 10 have work to do.
**
**     9 and 10 look up the exact TNUM in a table associated with the element
**        family to find the type, calling out to a GAP function to make each type
**        for the first time.
**
**     1 and 8 now get really complicated. This is because they now have to
**     check properties of the list which may be currently true, but not yet
**     known, and possibly not storable due to the presence of mutable
**     elements in the list. If we didn't do this, a lot of matrix stuff
**     wouldn't work
**
**     8 is the simpler. It calls KTnumHomPlist, which checks whether we
**     should really be in T_PLIST_CYC, T_PLIST_FFE or T_PLIST_TAB and if so,
**     changes the TNUM appropriately and returns the new tnum.  The only
**     time this is slow is a homogenous list of lists which looks like a
**     table until the very last entry which has the wrong length. This
**     should be rare.
**     
**     1 is the real nightmare, because it has to handle recursive mutable
**     lists, lists with mutable subobjects, etc.  We now concentrate on this
**     case.
**
**     The entry point is the function TypePlistWithKTnum, which returns both
**     the type and the ktnum of the list. This must be done in one function
**     to avoid an exponential slowdown for deeply nested lists. This
**     function is mutually recursive with KTnumPlist, which also returns two
**     pieces of information: the ktnum of the list and, if it is homogenous,
**     the family of the elements.
**
**     recursive lists (ie lists which are there own subobjects are detected
**     using the OBJ_FLAG_TESTING tnums. Any list being examined must have OBJ_FLAG_TESTING added to
**     its tnum BEFORE any element of it is examined.
**
**     
**
*/
\end{verbatim}

What is relevant for our purposes, is that this code, taking advantage
of the information already stored about the list, walks the structure
of it, as far as necessary to determine its ``transient'' TNUM (ie
the TNUM that it really deserves, but which might not be storable
thanks to the presence of mutable subobjects) storing as much
information along the way as it can. After that, the type is just
looked up in global variables for non-homogenous lists. For
homogeneous lists, the lookup is done in data stored in the lists
family. 

\section{Library Methods for List Arithmetic}

The main place where generic methods for list arithmetic are installed
seems to be in \verb|list.gi|. For Zero, methods are installed for any
dense small list, and for IsListDefault and
IsAdditiveElementWithZeroList (the kernel function \verb|ZERO_LIST_DEFAULT|,
which essentially just calls our old friend ZeroListDefault). Another
method checks dense lists to see if they are small but don't know it
yet. \textbf{Query:} why isn't the kernel method installed for any
dense small list, instead of a library method.

For Additive inverse, the story is similar, except that the kernel
method \textbf{is} installed for any dense small list.


For difference, various methods are installed which come back to the
kernel \verb|DIFF_LIST_SCL_DEFAULT| and \verb|DIFF_SCL_LIST_DEFAULT|,
which call the functions we have already seen. There are also some
library functions which handle non-dense lists.

For list difference, there is a library method which handles non-dense
lists and the kernel is called for appropriately types lists in
IsListDefault.

The kernel \verb|ONE_MATRIX| is installed for small ordinary matrices.

For inverse, there is \verb|INV_MATRIX| which makes no assumptions,
and \verb|INV_MAT_DEFAULT| which assumes plain lists.

There is a general installation of PROD as a method for POW for 
\verb|vector ^ matrix|.

Sum seems to parallel difference. 

A whole mass of method are installed for *. 

\section{How to Proceed?}

\subsection{The end of XTNums?}

An obvious question is whether the whole XTNum system really serves
any purpose.  If it were removed, then incomplete lists would fall
through to the library, and the type determination would do a
reasonable job of completing the TNum and storing appropriate
results. Indeed, it would probably do a better job than the XTNum
stuff because it has been worked on more recently. For instance the
XTNum of an immutable 2x2 integer matrix is incorrect. 

The cost is that method selection (presumably a cache hit) would be
needed to trigger the efficient kernel methods for cyclotomic and FFE
matrices. I think this is probably acceptable.

The whole virtual types machinery can then go.

This is implemented

\subsection{Implementing the New Rules}

So, we want to revise the existing kernel functions to implement the
left, right, ptwise and inner rules, as now specified. We need to
select an appropriate set of special cases, for which we might, for
instance require denseness, to have extra-fast routines. All of these
routines are exported from the kernel and installed in the library as
methods with appropriate requirements. 

Similarly, we need to revise existing library methods, both for their
semantics and their installation.

We also need some catch-all methods in the library that actually
compute nesting depths in difficult cases. 

We also need to check in which cases the TNUMs alone allow us to
choose one of the kernel methods, and install them in the jump tables.


\subsection{Another Thing}

More testing/programming is needed around fully mutable compressed
matrices, compressed locked vectors, etc.





\end{document}






