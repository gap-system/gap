\documentclass[12pt]{article}
\usepackage{a4wide}
\newcommand{\GAP}{\textsf{GAP}}
\newcommand{\bd}{\begin{description}}
\newcommand{\ed}{\end{description}}
\parindent=0pt
\parskip=\medskipamount
\title{The Road to \GAP\ 4.1 (and points beyond)}
\author{Steve Linton}
\begin{document}
\maketitle
\section{Introduction}

The purpose of this document is to identify all the remaining tasks
which must be performed before we can release \GAP\ version 4.1, so
that we can allocate, schedule and monitor them. It also includes some
ideas about developments not essential to 4.1, but which might be
nice, or which we might consider for release in
subsequent versions.

At this stage I am not considering tasks at the level of detail of
``fix bug x'' or the tasks involved in the actual release.

This document is based on discussions with Werner Nickel, Alexander
Hulpke, Frank Celler, Thomas Breuer, Volkmar Felsch and Joachim
Neub\"user in September 1997.

\section{Kernel tasks required for 4.1}

\bd
\item[Save/load workspace] partly implemented (SL), must be finished by 4.1 (SL to do).
\item[Fast vecffe arithmetic] in hand now (FC/TB) as part of a general overhaul of lists
in the kernel
\item[DOS and Mac ports] Must be done. Some code exists, but
interaction with streams etc will need work. Problems with some PC compilers.
\ed

\section{Library Packaging}

The library is now provisionally divided into packages:
\bd
\item[Essentials] Core things needed to keep \GAP\ running
\item[Algebra bootstrap] Definitions and naive methods for basic
algebraic domains
\item[Algebras] Lie, associative and generic algebras, includes ELIAS
\item[Abstract Groups] General methods, includes Morpheus
\item[PermGroups] Includes things like external sets
\item[FpGroups] Includes Tietze
\item[PcGroups] Groups using a PCGS
\item[Pcgs] computing and manipulating a Pcgs -- probably includes pQ
and nq
\item[Rewriting Systems] The boundaries around here aren't quite
settled
\item[Matrix Groups] 
\item[Character Tables basic] The electronic ATLAS
\item[Character Tables advanced] Interactive Character Table
Construction
\item[Finite Fields] Some things are in ``essentials''
\item[Number Theory] 
\item[Mappings]
\item[Vectorspaces]
\item[Polynomials]
\item[Tables of Marks]
\item[Compatibility]
\item[Data libraries] possibly one package per library
\ed

At this stage this is a purely administrative division. 

Someone should be assigned to oversee the maintenance of each package,
even in the run-up to 4.1, and surely afterwards. Many packages should
not need more than maintenance before 4.1.

\section{Library Tasks Required for 4.1}

\bd
\item[Quotient Algorithms] At a minimum, we need effective abelian
quotient and p-quotient code. As a last resort, a port of the
\texttt{anupq} package would do instead of a \GAP\ pq. At least an
architecture for, and preferably implementations of, nq and soluble
quotient would be nice.
\item[Tables of marks] Thomas Merkwitz is doing something here. As
with character tables, there is code to access and handle completed
tables, and there is code for interactively constructing them. The
latter is less urgent.
\item[Character Tables] The remaining work here is in hand. Thomas
Breuer will do it.
\item[2 and 3 group libraries] Frank Celler agreed to do this. At
least some of this is incorporated in Bettina's small groups library.
\item[Crystallographic Groups library] 
\item[Transitive Groups degrees 24 --31] If Alexander is happy to
release them
\item[Factorization over number fields] Alexander is doing
\item[Character degrees code for solvable groups] 
\item[Clifford matrices code] does this fall into the ``Character
tables'' work above
\item[Galois group determination] Alexander has plans for this
\item[General FP Group algorithms] Volkmar Felsch is working on
this. There is a lot to do, because of new concepts.
He will identify a suitable stopping point for 4.1.
\ed

\section{Share Packages}

Not, obviously entirely within our control. Clearly, we would like all
share packages upgraded to \GAP\ 4, but some are more important,
because they are central to some people's use of \GAP. In many cases,
functionality could be improved using \GAP\ 4 features. Maintainers
should probably be encouraged to port first and then improve in most
cases, in the interest of getting a \GAP\ 4 version out as soon as possible.

A list of \GAP\ 3 share packages, and what I know about their status:

\bd
\item[anupq] High priority. Single external program, so should be easy.
\item[anusq] Single external program, so should be easy.
\item[autag] Bettina is working in this area. Code exists in the \GAP
4 library, but may need final installation and tidying up.
\item[chevie] Medium Priority. Jean Michel and Frank L\"ubeck are certainly interested
in porting this
\item[cohomolo] 
\item[cryst] 
\item[dce]   
\item[gliss] In hand, I think.
\item[grape] High Priority. Len plans to do it after Christmas. Not
hard unless it is extended
\item[grim] Not hard
\item[guava] Medium Priority -- could be a problem (no maintainers)
\item[kbmag] Medium Priority -- shouldn't be a problem
\item[matrix] High Priority
\item[meataxe] Medium Priority unless kernel vecffe arithmetic is fast
enough. TB estimates it would take him 1 week
\item[monoid] Medium Priority (perhaps even absorb into the library)
\item[nq]     Medium Priority unless we have a \GAP\ nq by 4.1.
\item[pcqa]   Medium Priority
\item[sisyphos]
\item[specht]
\item[ve]     Medium Priority
\item[xgap]   See below
\item[xmod]   In hand
\ed

There are a couple of rumored forthcoming packages, such as the LBFM
sq.

\subsection{xgap} 

This breaks into several pieces. The core  xgap functionality and the
low-level \GAP\ functions to access it should be easily ported, and
should be done before \GAP\ 4.1. This may, in fact, already work.

There are various applications using xgap, to display and explore
subgroup lattices, tables of marks, submodule lattices and other
things. These need more thought: (a) because they may need to change
to match new concepts in the library and (b) because they grew up
rather haphazardly and should be organized better. This is not a key
priority for \GAP\ 4.1.

A Windows version exists, which worked under Windows NT 3.5, but has
stopped working under 4.0. Again not a 4.1 priority, but it must be
sorted eventually.


\section{Documentation}

There are four books in the current \GAP\ 4 manual. One or two books
of substantial worked examples may be added  (not a 4.1 priority).

\bd
\item[Users Tutorial] Some additional material is really needed here:
naming conventions \texttt{As} and \texttt{Is}, \texttt{Of} and
\texttt{By}, for example; some aspects of domains, possibly
more. Perhaps someone should be appointed ``editor'' and charged with
working out what is needed and then procuring and integrating it.

\item[Users Reference] The scheme for building this from skeleton
files and comments in the sources was basically agreed. This gives one
(rather urgent) task to implement that scheme and one task for each
package, to actually write (or convert from \GAP\ 3) the
documentation. Accuracy and completeness are more important here than
beautiful English composition.

\item[Programming Tutorial] A slightly lower priority than the user
documentation, but again an editor should be appointed. Use of the
compiler should be covered. Porting guidelines should be included.

\item[Programming Reference] This is essentially the same job as the
Users reference, but for certain parts of the ``Essentials''
package. Complicated by the rather elaborate rules that sometimes have
to be followed. The compiler interface should be documented.

\ed

\section{Kernel Issues not Essential for 4.1}
\bd

\item[A careful look at the break loop and debugging] There are still
a few bits of unexpected behaviour here, and more debugging support
could be added.
\item[Portability of compiled code] Some C compilers choke on the
output of the \GAP\ compiler. Can we avoid this?
\item[Windows NT Port] This will become increasingly necessary. See
also discussion of XGAP.
\item[Use of Autoconf or similar] This should simplify \GAP\ porting
and installation, especially if the C share packages can be made to
work with it as well.
\item[Making GASMAN malloc tolerant] This basically means enabling
operation with a non-contiguous workspace. There are a few issues. It
would simplify various other things though.
\item[Kernel support for marshalling] Marshalling is gathering all the
``pieces'' of an object for saving, loading, sending across the net,
etc. The problem is to avoid every object picking up the whole
library.
\item[Possible RPM package] .rpm and .deb packages would simplify
installation on popular flavours of Linux.
\item[Threads] Probably a \GAP 5 issue. Eventually a threaded kernel
would be nice.
\item[User Interface] Again a long-term issue, but there are a lot of
things that could be done here.
\ed

\section{Library Issues not essential for 4.1}

\bd
\item[Expression DAGs] an efficient representation for words in some
contexts
\item[Data libraries] New features in \GAP\ 4 might allow better and
more uniform presentation of data libraries
\item[FpGroups] This is a general area in need of review and update
\item[Beyond Lie algebras] A possible extension in the direction of
Kac-Moody algebras, quantum groups and related concepts would be of
considerable interest 
\item[(Transformation) Semigroups] There is a scope for a lot of
library development here.  
\item[MOC and meataxe] It is planned to try and incorporate most or
all of the functionality of MOC (somewhat urgent as it is very old
code) and the meataxe into \GAP. 
\ed 
\section{Schedule}

Frank will provisionally come to St Andrews in November (before I go
to Japan). Our goal at that time is to have the 4.1 kernel complete apart
form bug fixes, and (conceivably) additional algebraic functionality
moved into the kernel from elsewhere.  

Also by that time, it should be possible to finish nearly all the
essential library tasks, except perhaps the quotient algorithms
development.

Ideally beta 3 around Christmas should have the kernel and library
ready apart from bug fixes. Quotients is the most likely problem area here.

At this point it \emph{may} be appropriate to split off a 4.2 branch for
technical developments and merge in bug fixes from the 4.1 branch
after the release.

After that, there remains only documentation and key share packages,
plus any problems that arise. These will determine when 4.1 can
be released.

\end{document}
