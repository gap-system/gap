\documentclass[a4wide]{article}

\usepackage{amsmath}
\usepackage{theorem}
\usepackage{amssymb}
\usepackage{latexsym}
\usepackage{multirow}
\usepackage{a4wide}
\usepackage{alltt}

\newtheorem{thm}{Theorem}
\newtheorem{lem}[thm]{Lemma}
\newtheorem{prop}[thm]{Proposition}
\newtheorem{cor}[thm]{Corollary}
\newtheorem{df}[thm]{Definition}
\newtheorem{conj}[thm]{Conjecture}
{\theorembodyfont{\rmfamily} \newtheorem{exa}[thm]{Example}}

\newcommand{\ad}{\mathop{\mathrm{ad}}}
\newcommand{\Tr}{\mathop{\mathrm{Tr}}}
\newcommand{\Nr}{\mathop{\mathrm{NR}}}
\newcommand{\msl}{\begin{verbatim}\\end{verbatim}}

\newenvironment{rem}{\noindent{\bf Remark.}}{\newline}
\newenvironment{pf}{\noindent{\bf Proof.}}{\hfill $\Box$\newline}

\begin{document}

\title{The manual builder\\
experimental version 0.0}
\author{W. A. de Graaf\\
e-mail: {\tt wdg\@@dcs.st-and.ac.uk}}
\date{}
\maketitle

\section{Introduction}
This is an experimental version of a little document describing an
experimental version of the manual builder. Comments are heavily appreciated.

\section{The {\tt .gd} files}

The {\tt .gd} files contain most of the information that will go into the
manual.
First of all the {\tt .gd} file may contain blocks of text explaining general 
things (file headers). Every such block is required to start with {\tt \#}$n$,
where $n$ is some number. A block ends with an empty line.\par
Secondly, the {\tt .gd} file may contain declarations. It is assumed that
these have the following structure\par
\vspace{2mm}
{\tt
\noindent \#$\alpha$ Fct1( args ) \quad \quad comment\\
\#$\beta$ Fct2( args ) \quad \quad comment\\
.\\
.\\
\#$\zeta$ Fctn( args ) \quad \quad comment\\
\#\#\\
\#\#   explanatory text.....\\
\#\#\\
Fct1:= NewSomething(....);\\
.\\
.\\
Fctn:= NewSomething(....);
}
\par
\vspace{2mm}
Comments:
\begin{itemize}
\item $\alpha,\ldots,\zeta\in \{{\tt A},{\tt C},{\tt F},{\tt M},{\tt O},
{\tt P}\}$.
\item The letter of {\tt \#}$\alpha$ is printed in the manual. If this
letter is {\tt A} or {\tt P}, then also the letters {\tt S} and {\tt T}
are printed if the setter and the tester are available. If the letter
is {\tt A}, then the letter {\tt M} is printed if the attribute is mutable.
\item The comments on the same line as the {\tt Fct}-s will appear in
the manual, also on the same line; they are optional.
\item It is assumed that the {\tt \#}$\alpha$-s appear on consecutive lines.
(Is this too restrictive??)
\item After the lines starting with {\tt \#}$\alpha$ there is explanatory
text (possibly empty) on lines starting with {\tt \#\#}. Lines starting with
{\tt \#} + other character are ignored.
\item It is assumed that for each name {\tt Fcti} appearing in the set 
\{ {\tt Fct1,...,Fctn} \} there is a declaration of the form {\tt Fcti:=...}.
\end{itemize}

\begin{exa}
The file {\tt algebra.gi} contains the following declaration:\par
\vspace{2mm}\par
{\tt \noindent 
\#O  DirectSumOfAlgebras( <A1>, <A2> )...................for algebras\\
\#O  DirectSumOfAlgebras( <list> )\\
\#\#\\
\#\#  is the direct sum of the two algebras <A1> and <A2> resp. of the \\
\#\#  algebras in the list <list>.\\
\#\#\\
\#\#  If all involved algebras are associative algebras then the result \\
\#\#  is also known to be associative.\\
\#\#  If all involved algebras are Lie algebras then the result is \\
\#\#  also known to be a Lie algebra.\\
\#\#\\
\#\#  All involved algebras must have the same left acting domain.\\
\#\#\\
\#\#  The default case is that the result is a s.c. algebra.\\
\#\#\\
\#\#  If all involved algebras are matrix algebras, and either both are\\
\#\#  Lie algebras or both are associative then the result is again a\\
\#\#  matrix algebra of the appropriate type.\\
\#\#\\
DirectSumOfAlgebras := NewOperation( "DirectSumOfAlgebras",\\
    {}[ IsDenseList ] );
}\par
\vspace{2mm}\par
This will lead to the following item in the manual:\par
\vspace{2mm}
{\noindent \bf \large 0.1 DirectSumOfAlgebras}
\vspace{2mm}\par
\begin{tabular}{llr}
{\tt DirectSumOfAlgebras( {\it A1}, {\it A} )} & O & for algebras\\
{\tt DirectSumOfAlgebras( {\it list} )} & O & \\
\end{tabular}
\vspace{3mm}


\noindent is the direct sum of the two algebras {\it A1} and {\it A2} resp. 
of the algebras  in the list {\it list}. 
If all involved algebras are associative algebras then the result is also 
known to be associative. 
If all involved algebras are Lie algebras then the result is also known 
to be a Lie algebra. 
All involved algebras must have the same left acting domain. 
The default case is that the result is a s.c. algebra. 
If all involved algebras are matrix algebras, and either both are Lie 
algebras or both are associative then the result is again a 
matrix algebra of the appropriate type. 
\end{exa}

\section{The {\tt .gi} and {\tt .g} files}

The {\tt .gi} and {\tt .g} files may contain comments starting with {\tt \#+}. 
All such comments following a line starting with {\tt \#}$\alpha$ are grouped 
togeter and will appear in the manual in the section with the title 
corresponding to the declaration on the {\tt \#}$\alpha$ line.\par
Furthermore the {\tt .gi} files may contain declarations that will go into the
manual. If a method is installed by {\tt InstallOtherMethod} {\em and}
this method has comments ({\tt \#+}) in its body {\em and} it has a
declaration (on a line starting with {\tt \#M}), then this declaration
will be added to the declaration lines from the {\tt .gd} file. In the
comments appearing in the body the string {\tt NUMBER} will be replaced 
by the natural number reflecting the position of the declaration in the
list of all declarations. It is assumed that the header contains at least
as many {\tt \#M} declarations as there are {\tt InstallMethod} or
{\tt InstallOtherMethod} lines. The $k$-th {\tt Install(Other)Method} line
belongs to the $k$-th {\tt \#M} line.

\section{The {\tt .msk} files}

The {\tt .msk} files contain the commands that decide what goes where in the
manual. Currently the following commands are supported:
\begin{itemize}
\item {\tt \verb!\!FileHeader[nr]\{filename\}} the file header 
starting with {\tt \#nr} of the file {\tt filename.gd}. 
If {\tt [nr]} is ommitted then {\tt nr} is assumed to be 1.
All fileheaders are assumed to end with an empty line (so a line without
even a {\tt \#}). 
\item {\tt \verb!\!Declaration\{Gizmo\}} creates a section for {\tt Gizmo}.
\item {\tt \verb!\!beginexample} begins an example.
\item {\tt \verb!\!endexample} ends an example. All text appearing on the
lines following the one containing {\tt \verb!\!beginexample} and prior to
the one containing {\tt \verb!\!endexample} is printed in {\tt tt}-format.
\item {\tt \verb!\!Implications[lev] } (for categories). Here {\tt lev} 
is an integer, if it is omitted, then it is assumed to be 1. {\tt lev} can be 
one of the following
\begin{itemize}
\item[{\tt 1}] all categories implied by {\tt Gizmo} are listed.
\item[{\tt 2}] all categories that imply {\tt Gizmo} are listed.
\end{itemize}

\item {\tt \verb!\!Requirements} this lists the requirements to the 
arguments as contained in the {\tt .gd} file.
\item {\tt \verb!\!Methods} this lists the argument-requirements for which 
methods are installed in the {\tt .gi} file. 
\end{itemize}

\section{Using the program}

The program loops over a two lists of filenames. The first list contains
the {\tt .gd} files and the second list the {\tt .msk} files. 
A tex file is created for each entry of the list of {\tt .msk} files.
If, for example, the files {\tt exa1.gd}, {\tt exa2.gd}, {\tt exa3.gd}, 
{\tt exa1.msk} and {\tt exa2.msk} have been created, then the third
line of the manual builder has to be changed into {\tt \@@gdfiles = ("exa1",
"exa2","exa3");} and the fourth line has to be changed into 
{\tt \@@msfiles = ("exa1","exa2");}.
By invoking the program the files {\tt exa1.tex} and {\tt exa2.tex} will be 
created. The program
checks whether the files {\tt exa*.gi} and {\tt exa*.g} exist and if so reads
information from them.\par
It is also posssible to put the information about the files into a config
file. Such a file should look like the following:\\
\vspace{2mm}\\
{\tt 
\noindent @msfiles = ("exa","exa1");\\
@gdfiles = ("exa","exa1");\\
\$DIR = "/home/wdg/perl/try";\\
}

The {\tt \$DIR} variable specifies a directory for the program to write in
(it can be omitted in which case the current directory is used). Supposing
that the above file is called {\tt config}, the following command will
work:\\
\vspace{2mm}\\
{\tt \noindent wdg.cochran > buildman.pe -f config\\
Reading files...\\
Composing the TEX file exa.tex\\
Composing the TEX file exa1.tex\\
}

\section{Variable replacement}

The configuration file may also contain lines that assign variables
\begin{verbatim}
gapname=4r1
datum=5. Nov. 1837
\end{verbatim}

Occurences of these variables in double squiggly brackets will be replaced
by their value. For example a line
\begin{verbatim}
The last release is {{gapname}} of {{datum}}
\end{verbatim}
will be replaced by:
\begin{verbatim}
The last release is 4r1 of 5. Nov. 1837
\end{verbatim}
This feature is very handy for information that changes over time.

\end{document}
















