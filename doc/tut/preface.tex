%%%%%%%%%%%%%%%%%%%%%%%%%%%%%%%%%%%%%%%%%%%%%%%%%%%%%%%%%%%%%%%%%%%%%%%%%
%%
%W  preface.tex               GAP documentation         Joachim Neubueser
%%
%H  @(#)$Id$
%%
%Y  Copyright (C) 1997, Lehrstuhl D fuer Mathematik, RWTH Aachen, Germany
%%
%%  This file contains the preface of the GAP manual.
%%
%%%%%%%%%%%%%%%%%%%%%%%%%%%%%%%%%%%%%%%%%%%%%%%%%%%%%%%%%%%%%%%%%%%%%%%%%
\Chapter{Preface}

Welcome to {\GAP}. This preface serves not only to introduce this
manual, ``the {\GAP} Tutorial'', but also as an introduction to the
system as a whole, and in particular to changes between the current and
previous versions.

{\GAP} stands  for  *Groups, Algorithms  and Programming*.  The  name was
chosen to reflect  the aim of  the  system, which  is introduced in  this
tutorial manual.  Since  that  choice,  the  system has become   somewhat
broader,    and  you will  also   find information   about algorithms and
programming  for other   algebraic structures,   such as semigroups   and
algebras.

There are four further manuals in addition to  this one: the ``Reference
Manual''    containing   detailed  documentation     of  the mathematical
functionality   of {\GAP}; ``Extending   GAP''  containing  some tutorial
material on various aspects  of {\GAP} programming;  ``Programming in
GAP 4'' containing detailed documentation   of various aspects   of the
system of interest mainly to programmers; and ``New Features for
Developers'' containing details of some newly introduced features
which we may wish to change in a future release and so do not want to
include in the main reference manual. Some of the functionality
of the system and a number of contributed extensions are provided as
``{\GAP} packages'' and each of these has its own manual. This
preface, however, serves as an introduction to the whole system.

Subsequent sections of this preface explain the structure of the
system and the arrangements for the attribution of credit for
authorship and maintenance of the system; acknowledge those who have
made particular contributions to this and previous releases and
outline the changes from earlier versions.



%%%%%%%%%%%%%%%%%%%%%%%%%%%%%%%%%%%%%%%%%%%%%%%%%%%%%%%%%%%%%%%%%%%%%%%%
\Section{The GAP System}


{\GAP}  is  a  *free*, *open*  and    *extensible*  software package  for
computation  in  discrete abstract  algebra.    The terms ``free''  and
``open'' describe the conditions under which the system is distributed --
in brief, it is *free of charge* (except possibly for the immediate costs
of delivering it to  you), you are *free  to pass  it on* within  certain
limits,  and all  of the  workings of  the  system are  *open  for you to
examine and   change*. Details of these conditions can be found
%display{tex}
in the Copyright Notice of the previous page.
%display{nontex}
%in "ref:Copyright".
%enddisplay

The system is ``extensible'' in that you can write your own  programs  in
the {\GAP} language, and use them in just the same way  as  the  programs
which form part of the system  (the  ``library'').  Indeed,  we  actively
support the contribution, refereeing and distribution  of  extensions  to
the system, in the form of ``{\GAP} packages''.  Further details of  this
can be found in chapter "ref:GAP Packages" in the Reference Manual, and
on our World Wide Web site.

Development of {\GAP} began at Lehrstuhl D f\accent127ur Mathematik,
RWTH-Aachen, under the leadership of Joachim Neub\accent127user
in 1985. Version 2.4 was released in 1988 and version 3.1 in 1992.
In 1997 coordination of 
{\GAP} development, now very much an international effort, was
transferred to St Andrews.  A complete internal redesign and almost
complete rewrite of the system was completed over the following years and
version 4.1 was released in July 1999.

A sign of the further internationalization of the project is this release,
4.4, which has been coordinated from Colorado State University, Fort Collins.

More information on the motivation and development of {\GAP} to date,
can be found on our Web pages in a section entitled ``Release history
and Prefaces''.

For those readers who have used an earlier version of {\GAP}, an
overview of the changes from {\GAP}~4.3, and a brief summary of
changes from earlier versions is given in section "Changes from Earlier
Versions" below.

The system that you are getting now consists of a ``core system'' and
a number of packages. The core system consists of four main parts.
\beginlist%ordered
  \item{1.}
    A  *kernel*, written in C, which provides the user with
    \itemitem{--}%unordered
      automatic dynamic storage management, which the user needn't bother
      about in his programming;
    \itemitem{--}
      a   set of  time-critical basic   functions, e.g.   ``arithmetic'',
      operations for integers, finite fields,  permutations and words, as
      well as natural operations for lists and records;
    \itemitem{--} 
      an interpreter   for  the {\GAP} language,    an untyped
      imperative programming language with functions as first class objects
      and some extra built-in data types such as permutations and finite
      field elements.  The language supports a form of object-oriented
      programming, similar to that supported by languages like C++ and
      Java but with some important differences.
    \itemitem{--}
      a small set of system functions allowing the {\GAP} programmer to handle
      files and execute external programs in a uniform way, regardless of 
      the particular operating system in use. 
    \itemitem{--}
      a  set  of programming tools  for   testing, debugging, and timing
      algorithms.
    \itemitem{--}
      a ``read-eval-view'' style user interface.

  \item{2.}  
	A much larger *library of {\GAP} functions* that
  implement algebraic and other algorithms.  Since this is written
  entirely in the {\GAP} language, the {\GAP} language is both the
  main implementation language and the user language of the system.
  Therefore the user can as easily as the original programmers
  investigate and vary algorithms of the library and add new ones to
  it, first for own use and eventually for the benefit of all {\GAP}
  users.

  \item{3.}  
	A *library of group theoretical data* which contains
  various libraries of groups, including the library of small groups
  (containing all groups of order at most 2000, except those of order
  1024) and others. Large libraries of ordinary and Brauer character
  tables and Tables of Marks are included as packages.

  \item{4.}
    The *documentation*.  This is available as on-line help, as
    printable files in various formats and as HTML for viewing
    with a Web browser.

\endlist

Also included with the core system are some test files and a few
small utilities which we hope you will find useful.


*{\GAP} packages* are self-contained extensions to the core system.  A
package contains {\GAP} code and its own documentation and may also
contain data files or external programs to which the {\GAP} code
provides an interface.  These packages may be loaded into {\GAP} using
the `LoadPackage' command, and both the package and its
documentation are then available just as if they were parts of the
core system. Some packages may be loaded automatically, when {\GAP} is
started, if they are present. Some packages, because they depend on
external programs, may only be available on the operating systems
where those programs are available (usually UNIX). You should note
that, while the packages included with this release are the most
recent versions ready for release at this time, new packages and new
versions may be released at any time and can be easily installed in
your copy of {\GAP}.

With {\GAP} there are two packages (the library of ordinary and
Brauer character tables, and the library of tables of marks) which
contain functionality developed from parts of the {\GAP} core
system. These have been moved into packages for ease of maintenance
and to allow new versions to be released independently of new releases
of the core system. (For technical reasons the library of tables of marks is
still distributed in the main system archive.) 
The library of small groups should also be
regarded as a package, although it does not currently use the standard
package mechanism.  Other packages contain functionality which has
never been part of the core system.

%%%%%%%%%%%%%%%%%%%%%%%%%%%%%%%%%%%%%%%%%%%%%%%%%%%%%%%%%%%%%%%%%%%%%%%
\Section{Authorship and Maintenance}

Previous versions of {\GAP} have simply included the increasingly long
list of all of the authors of the system with no indication as to who
contributed what. With {\GAP} 4.3 we have introduced a new concept:
*modules*, to allow us to report the authorship of the system in more
detail. A module is a part of {\GAP} which provides identifiable
functionality and has reasonably clean interfaces with the rest of the
system (usually it consists of separate files). Each module has its
own lists of authors and maintainers, which are not necessarily the
same. A preliminary list of modules and their attributions appears in
this manual. Note that we are still in the process of identifying
modules within the system, so large parts of the system do not yet
fall into any module. Since also we all collaborate closely in
designing, developing and debugging the system, it should not be
assumed that the list of modules in this manual represents all of
everyone's contribution, or that it lists everyone who made any
contribution at all to each module.

All {\GAP} packages are also considered to be modules and have their
own authors and maintainers. It should however be noted that some
packages provide interfaces between {\GAP} and an external program, a
copy of which is included for convenience, and that, in these cases,
we do not claim that the module authors or maintainers wrote, or
maintain, this external program. Similarly, some modules and packages
include large data libraries that may have been computed by many
people. We try to make clear in each case what credit is attributable
to whom.

We have, for some time, operated a refereeing system for contributed
packages, both to ensure the quality of the software we distribute,
and to provide recognition for the authors. We now consider this to be
a refereeing system for modules, and we would note, in particular
that, although it does not use the standard package interface, the
library of small groups has been refereed and accepted on exactly the
same basis as the accepted  packages.

We also include with this distribution a
number of packages which have not (yet) gone through our refereeing
process. Some may be accepted in the future, in other cases the
authors have chosen not to submit them.  More information can be found
on our World Wide Web site, see section "Further Information about
GAP".


%%%%%%%%%%%%%%%%%%%%%%%%%%%%%%%%%%%%%%%%%%%%%%%%%%%%%%%%%%%%%%%%%%%%%%%
\Section{Acknowledgements}

Very many people have worked on, and contributed to, {\GAP} over the
years since its inception. On our Web site you will find the prefaces
to the previous releases, each of which acknowledges people who have
made special contributions to that release. Even so, it is appropriate
to mention here Joachim Neub\accent127user whose vision of a free,
open and extensible system for computational algebra inspired {\GAP}
in the first place, and Martin Sch\accent127onert, who was the
technical architect of {\GAP} 3 and {\GAP} 4.

In the past years {\GAP} development
has become a more and more widely distributed operation, and
increasingly dependent on hard voluntary work by developers not solely
or mainly employed to work on {\GAP}. 

Nevertheless, the development
process has remained constructive and friendly, even when wrangling
over difficult technical decisions, or sensitive questions of
attribution and credit and I must express our huge gratitude to
everyone involved for this.

The list of modules which appears in this manual now gives a partial
idea of the contributions of different people, but we would like to
mention some people who have made important contributions to the
development process over the last years that do not show up there:

Steve Linton has been leading the {\GAP} group in St Andrews over the last
years and continues to be the main kernel maintainer and developer. The group
in St Andrews also maintains most of the development infrastructure and 
helps with user support.

Thomas Breuer continues to develop many areas of the system, and
to play a vital role in clarifying our underlying concepts, despite
now working in industry.

Frank L\accent127ubeck set up a new system for automatic handling of
packages and helped with various kernel issues.

Bettina Eick and her research group in Braunschweig have contributed much
functionality, mainly in the form of packages.

Max Neunh\accent127offer has brought much fresh insight to bear on the
design of crucial parts of the system, and also done a lot of the
ensuing work; Stefan Kohl and Volkmar Felsch have both brought
enormous persistence to pointing out errors and inconsistencies in code and 
documentation, improving error messages and generally polishing the
system; and very many others have contributed ideas, insight and hard
work to produce this release. Senior colleagues, especially Joachim
Neub\accent127user, Edmund Robertson, and Charley Wright, continue to
provide encouragement support and constructive criticism.



%%%%%%%%%%%%%%%%%%%%%%%%%%%%%%%%%%%%%%%%%%%%%%%%%%%%%%%%%%%%%%%%%%%%%%%%%
\Section{Changes from Earlier Versions}

The main changes between {\GAP} 4.3 and {\GAP} 4.4 are:

*Potentially Incompatible Changes*

\beginlist%unordered

\item{--}
  The mechanism for the loading of Packages has changed to allow easier
  updates independent of main {\GAP} releases. Packages require a
  file `PackageInfo.g' now. The new `PackageInfo.g' files are available for all
  Packages with the new version of GAP.

\item{--}
  `IsSimple' returns false now for the trivial group.

\item{--}
  `PrimeBlocks': The output format has changed.

\item{--}
  `Division Rings': These are implemented as `IsRingWithOne' now.

\item{--}
  `DirectSumOfAlgebras': $p$-th power maps are compatible with the input now.

\item{--}
  The print order for polynomials has been changed.
\endlist
These changes are, in some respects, departures from our policy of
maintaining upward compatibility of documented functions between
releases. In the first case, we felt that the old behavior was
sufficiently inconsistent, illogical, and impossible to document that
we had no alternative but to change it. In the case of the package
interface, the change
was necessary to introduce new functionality. The planned and phased
removal of a few unnecessary functions or synonyms is needed to avoid
becoming buried in ``legacy'' interfaces, but we remain committed to
our policy of maintaining upward compatibility whenever sensibly possible.


\beginlist%unordered

\item{--}
  Groebner Bases:

    Buchberger's algorithm to compute Groebner Bases has been implemented
    in GAP. (A. Hulpke)

\item{--}
    For large scale Groebner Basis computations there also is an interface
    to the Singular system available in the `Singular' Package.
    (M. Costantini and W. de Graaf)

\item{--}
    New methods for factorizing polynomials over algebraic extensions of
    the rationals have been implemented in GAP. (A. Hulpke)

\item{--}
    For more functionality to compute with algebraic number fields there
    is an interface to the Kant system available in the `Alnuth' Package.
    (B. Assmann and B. Eick)

\item{--}
    A new functionality to compute the minimal normal subgroups
    of a finite group, as well as its socle, has been installed. (B.
    H\accent127ofling)

\item{--}
    A fast method for recognizing whether a permutation group is symmetric
    or alternating is available now (A. Seress)

\item{--}
    A method for computing the Galois group of a rational polynomial is
    available again.  (A. Hulpke)
    
\item{--}
    The algorithm for `BrauerCharacterValue' has been extended to the case
    where the splitting field is not supported in GAP. (T. Breuer)

\item{--}
    Brauer tables of direct products can now be constructed from the
    known Brauer tables of the direct factors. (T. Breuer)

\item{--}
    Basic support for vector spaces of rational functions and of uea
    elements is available now in GAP. (T. Breuer and W. de Graaf)

\item{--}
    Various new functions for computations with integer matrices are
    available, such as methods for computing normal forms of integer
    matrices as well as nullspaces or solutions systems of equations.
    (W. Nickel and F. Gaehler)
\endlist

*New Packages*

  The following new Packages have been accepted.

\beginlist%unordered

\item{--}
   Alnuth: Algebraic Number Theory and an interface to the Kant system.
    By B. Assmann and B. Eick

\item{--}
   LAGUNA: Computing with Lie Algebras and Units of Group Algebras.
    By V. Bovdi, A. Konovalov, R. Rossmanith, C. Schneider.

\item{--}
   NQ: The ANU Nilpotent Quotient Algorithm.
    By W. Nickel.

\item{--}
   KBMAG: Knuth-Bendix for Monoids and Groups.
    By D. Holt.

\item{--}
   Polycyclic: Computation with polycyclic groups.
    By B. Eick and W. Nickel

\item{--}
   QuaGroup: Computing with Quantized Enveloping Algebras.
    By W. de Graaf.
\endlist

*Performance Enhancements*

\beginlist%unordered
\item{--}
    The computation of irreducible representations and irreducible
    characters using the Baum-Clausen algorithm and the implementation of
    the Dixon-Schneider algorithm have been speeded up.
    
\item{--}
    The algorithm for
    `PossibleClassFusions' has been changed: the efficiency is improved and
    a new criterion is used. The algorithm for `PossibleFusionsCharTableTom'
    has been speeded up. The method for `PrimeBlocks' has been improved
    following a suggestion of H. Pahlings.

\item{--}
    New improved methods for normalizer and subgroup conjugation in Sn have
    been installed and new improved methods for IsNaturalSn/An have been
    implemented. These improve the available methods when groups of large
    degrees are given.

\item{--}
    The partition split method used in the permutation backtrack is now
    in the kernel. Transversal computations in large permutation groups
    are improved. Homomorphisms from free groups into permutation groups
    now give substantially shorter words for preimages.

\item{--}
    The membership test in SP and SU has been improved using the invariant
    forms underlying these groups.

\item{--}
    An improvement for the cyclic extension method for the computation of
    subgroup lattices has been implemented.

\item{--}
    A better method for MinimalPolynomial for finite field matrices has been
    implemented.

\item{--}
    The display has changed and the arithmetic of multivariate polynomials
    has been improved.

\item{--}
    The `LogMod' function
    now uses Pollard's rho method combined with the Pohlig/Hellmann approach.

\item{--}
    Various functions for sets and lists have been improved following
    suggestions of L. Teirlinck. These include: Sort, Sortex, SortParallel,
    SortingPerm, NrArrangements.

\item{--}
    The methods for `StructureConstantsTable' and `GapInputSCTable' have
    been improved in the case of a known (anti-) symmetry following a
    suggestion of M. Costantini.
\endlist

The improvements listed in this Section have been implemented by T. Breuer
and A. Hulpke.

*New Programming and User Features*

\beginlist%unordered
\item{--}
The 2GB limit for workspace size has been removed and version numbers for
saved workspaces have been introduced. (S. Linton and B. H\accent127ofling)

\item{--}
  The limit on the total number of types created in a session
           has been removed. (S. Linton)

\item{--}
  There is a new mechanism for loading packages available.
              Packages need a file PackageInfo.g now. (T. Breuer and
              F. L\accent127ubeck)
\endlist

Finally, as always, a number of bugs have been fixed. This release thus incorporates the contents
of all the bug fixes which were released for {\GAP} 4.3. It also fixes
a number of bugs discovered since the last bug fix.
\bigskip

The most important changes between {\GAP} 4.2 and {\GAP} 4.3 were:

\beginlist%unordered

\item{--}
The performance of several routines has been substantially improved.

\item{--}
The functionality in the areas of finitely presented groups, Schur covers
and the calculation of Representations has been extended.

\item{--}
The data libraries of transitive groups, finite integral matrix groups,
character tables and tables of marks have been extended.

\item{--}
The Windows installation has been simplified for the case where you
are installing GAP in its standard location.

\item{--}
Many bugs have been fixed.

\endlist


The most important changes between {\GAP} 4.1 and {\GAP} 4.2 were:

\beginlist%unordered
\item{--}
A much extended and improved library of small groups as well as
associated IdGroup routines.

\item{--}
The primitive groups library has been made more independent of the
rest of GAP, some errors were corrected.

\item{--}
New (and often much faster) infrastructure for orbit computation, based on a
general ``dictionary'' abstraction.

\item{--}
New functionality for dealing with representations of algebras, and
in particular for semisimple Lie algebras. 

\item{--}
New functionality for binary relations on arbitrary sets, magmas and
semigroups.

\item{--}
Bidirectional streams, allowing an external process to be started and then
controlled ``interactively'' by GAP 

\item{--}
A prototype implementation of algorithms using general subgroup chains.

\item{--}
Changes in the behavior of vectors over small finite fields.

\item{--}
A fifth book ``New features for Developers'' has been added to the GAP manual.

\item{--}
Numerous bug fixes and performance improvements 
\endlist



The changes between the final release of {\GAP} 3 (version 3.4.4) and
{\GAP} 4 are  wide-ranging.  The general philosophy of the
changes is two-fold.  Firstly, many assumptions in the design of
{\GAP} 3 revealed its authors' primary interest in group theory, and
indeed in finite group theory. Although much of the {\GAP} 4 library
is concerned with groups, the basic design now allows extension to
other algebraic structures, as witnessed by the inclusion of
substantial bodies of algorithms for computation with semigroups and
Lie algebras.  Secondly, as the scale of the system, and the number of
people using and contributing to it has grown, some aspects of the
underlying system have proved to be restricting, and these have been
improved as part of comprehensive re-engineering of the system. This
has included the new method selection system, which underpins the
library, and a new, much more flexible, {\GAP} package interface.

Details of these changes can be found in chapter "Migrating to GAP 4" of
this manual. It is perhaps worth mentioning a few points here. 

Firstly, much remains unchanged, from the perspective of the mathematical 
user:

\beginlist%unordered
  \item{--}
    The syntax of that part  of the {\GAP} language  that most users need
    for investigating mathematical problems.

  \item{--}
    The great majority of function names.

  \item{--}
    Data libraries and the access to them.
\endlist

A number of visible aspects have changed:

\beginlist%unordered
  \item{--}
    Some function names that need finer specifications now that there are
    more structures available in {\GAP}.

  \item{--}  
    The    access to information   already  obtained about a mathematical
    structure. In {\GAP}~3 such information about a group could be looked
    up  by  directly inspecting  the  group record,  whereas in  {\GAP}~4
    functions must be used to access such information.
\endlist

Behind the scenes, much has changed: 

\beginlist%unordered
  \item{--} A new kernel,  with improvements in  memory management  and in
  the language interpreter, as well  as new  features  such as saving  of
  workspaces and the possibility of compilation of {\GAP} code into C.

  \item{--} A new structure   to the library, based  upon  a new  type and
  method  selection system, which  is able to  support a broader range of
  algebraic computation and to make the  structure of the library simpler
  and more modular.

  \item{--}
    New and faster algorithms in many mathematical areas.

  \item{--} 
    Data structures and algorithms for new mathematical objects, such as
    algebras and semigroups.

  \item{--}
    A new and more flexible structure  for the {\GAP} installation
    and documentation, which  means, for example, that a {\GAP} package and
    its documentation can be installed and be fully usable without any changes
    to the {\GAP} system.
\endlist

Very few features of {\GAP}~3 are not yet available in  {\GAP}~4.

\beginlist%unordered
   \item{--}
     Not all of the {\GAP}~3 packages have yet been converted
     for use with  {\GAP}~4 (although several new packages are available
     only in  {\GAP}~4). 

   \item{--} The library of crystallographic groups which was present in
     {\GAP}~3 is now part of a {\GAP}~4 package `crystcat'.

\endlist


%%%%%%%%%%%%%%%%%%%%%%%%%%%%%%%%%%%%%%%%%%%%%%%%%%%%%%%%%%%%%%%%%%%%%%%%%
\Section{Further Information about GAP}

\atindex{web sites!for GAP}{@web sites!for GAP}
Information about {\GAP} is best obtained from the {\GAP} Web pages that
you find on:

\URL{http://www.gap-system.org}

and its mirror at:

\URL{http://www.math.rwth-aachen.de/~GAP}

There you will find, amongst other things
\beginlist%unordered
\item{--} directions to the sites from which you can download the
current {\GAP} distribution, any bug-fixes, all accepted {\GAP} packages,
and a selection of other contributions.
\item{--} the {\GAP} manual and an archive of the `gap-forum' mailing
list, formatted for reading with a Web browser, and indexed for
searching.
\item{--} information about {\GAP} developers, and about the email
addresses available for comment, discussion and support.
\endlist

\index{email addresses}\indextt{support!email address}

We would particularly ask you to note the following things:
\beginlist%unordered
\item{--} The  {\GAP} Forum -- an email discussion forum for comments,
discussions or questions about {\GAP}. You must subscribe to the list
before you can post to it, see the Web page for details.
In  particular we will announce bugfixes in this mailing list.
\item{--} The email address \Mailto{support@gap-system.org} to which you
are asked to send any questions or bug reports which do not seem likely
to be of interest to the whole {\GAP} Forum. Section~"ref:If Things Go Wrong"
in the Reference Manual tells you what to include in a bug report.
\item{--} We also ask you to send a brief message to \Mailto{support@gap-system.org} when you install {\GAP}.
\item{--} The correct form of citation of {\GAP}, which we ask you use
whenever you publish scientific results obtained using {\GAP}.
\endlist

It finally remains  for me to wish you  all pleasure and success in using
{\GAP}, and to invite your constructive comment and criticism.
\bigskip\bigskip

Fort Collins, March 2004 \hfill Alexander Hulpke


%%%%%%%%%%%%%%%%%%%%%%%%%%%%%%%%%%%%%%%%%%%%%%%%%%%%%%%%%%%%%%%%%%%%%%%%%
%%
%E  preface.tex . . . . . . . . . . . . . . . . . . . . . . . . ends here
