%%%%%%%%%%%%%%%%%%%%%%%%%%%%%%%%%%%%%%%%%%%%%%%%%%%%%%%%%%%%%%%%%%%%%%%%%
%%
%W  opers.tex                 GAP documentation            Heiko Theissen
%%
%H  @(#)$Id$
%%
%Y  Copyright 1997,  Lehrstuhl D fuer Mathematik,  RWTH Aachen,   Germany
%%
%%  This file contains a tutorial introduction to operations.
%%


%%%%%%%%%%%%%%%%%%%%%%%%%%%%%%%%%%%%%%%%%%%%%%%%%%%%%%%%%%%%%%%%%%%%%%%%%
\Chapter{Operations and Methods}

%%%%%%%%%%%%%%%%%%%%%%%%%%%%%%%%%%%%%%%%%%%%%%%%%%%%%%%%%%%%%%%%%%%%%%%%%
\Section{Attributes}

In the  preceding chapters, we have  seen how to obtain information about
mathematical objects in {\GAP}: We have to pass the object as an argument
to a function. For example, if <G> is a group one can call
`Size( <G> )', and the function will return a value,
in our example an integer  which is the  size of <G>. Computing
the size  of a  group generally  requires  a substantial amount  of work,
therefore it seems desirable to store the size somewhere once it has been
calculated. You should imagine that {\GAP} stores  the size in some place
associated  with the object  <G> when `Size( <G>  )'  is executed for the
first time, and if this function  call is executed  again later, the size
is simply looked up and returned, without further computation.

\index{getter!of an attribute}\index{setter!of an attribute}%
\index{tester!of an attribute}\index{methods}%
This means that the  behavior  of the  function  `Size' has to  depend on
whether the size for the argument <G> is already known,  and if not, that
the size  must be  stored after it  has been  calculated. These two extra
tasks are  done  by two other   functions that accompany `Size(   <G> )',
namely the *tester* `HasSize( <G> )'
and the *setter* `SetSize( <G>, <size> )'.
The tester returns `true' or `false' according to
whether <G> has already stored its size, and the setter puts <size> into
a place from where <G> can directly look it up.
The function `Size' itself is called the *getter*,
and from the preceding discussion we see
that there must really be at least two *methods* for the getter:
One method is used when the tester returns `false';
it is the method which first does the real computation and then executes
the setter with the computed value.
A second method is used when the tester returns `true';
it simply returns the stored value.
This second method is also called the *system getter*.
{\GAP} functions for which several methods can be available
are called *operations*, so `Size' is an example of an operation.
\begintt
gap> G := Group( (1,2,3,4,5,6,7,8), (1,2) );; Size( G ); time;
40320
90   # this may of course vary on your machine
gap> Size( G ); time;
40320
0
\endtt
The convenient thing  for the user  is that  {\GAP} automatically chooses
the right method  for the getter, i.e.,  it calls a real-work getter at
most once  and the system getter  in all subsequent occurrences. *At most
once* because the value of a function call like `Size( <G> )' can also be
set for <G>  before the getter  is called at all;
for example, one can call the setter directly if one knows the size.

The size of a group is an example of a class of things  which in {\GAP}
are called *attributes*.
Every attribute in  {\GAP} is represented by  a triple  of a getter,  a
setter   and a  tester.  When a  new  attribute  is  declared, all  three
functions are created together and  the getter contains references to the
other two.  This is necessary because when  the getter is called, it must
first  consult the tester,   and perhaps execute the  setter  in the end.
Therefore the getter could be implemented as follows:
\begintt
getter := function( obj )
local   value;
    if tester( obj )  then
        value := system_getter( obj );
    else
        value := real_work_getter( obj );
        setter( obj, value );
    fi;
    return value;
end;
\endtt
The  only  function which  depends on   the  mathematical  nature  of the
attribute  is  the  real-work getter,  and this   is  of course  what the
programmer of  an  attribute has to  install.  In both cases,  the getter
returns  the same value, which  we also call  the  value of the attribute
(properly: the value of the attribute for the object `obj').
By the way:
The names for setter and tester of an attribute are always composed from
the prefix `Set' resp.~`Has' and the name of the getter.

As a (not typical) example, note that the {\GAP} function `Random',
although it takes only one argument, is of course *not* an attribute,
because otherwise the first random element of a group would be stored by
the setter and returned over and over again by the system getter
every time `Random' is called in the sequel.)

There is a general important rule about attributes: *Once the value of an
attribute for an object has been set, it cannot be reset, i.e., it cannot
be changed any more.* This is achieved by having two methods not only for
the getter but also for the setter: If an object already has an attribute
value stored, i.e., if the tester  returns `true', the setter simply does
nothing.
\beginexample
gap> G := SymmetricGroup(8);; Size(G);
40320
gap> SetSize( G, 0 ); Size( G );
40320
\endexample

%\exercise Experiment  with {\GAP} to  find out whether immutability of an
%object prevents a setter from setting a previously unset value.
%
%\answer It does  not. That an  object like a  group is constant  does not
%mean  that   additional information cannot    be entered. Such additional
%information does not change the mathematical identity of the object.

{\bf  Summary.} In this section  we have introduced attributes as triples
of getter, setter   and tester and    we have explained how  these  three
functions work together behind  the  scene to provide automatic  storage
and look-up of  values that have once been  calculated. We have seen that
there can be several methods for  the   same function  among which {\GAP}
automatically selects an appropriate one.

%%%%%%%%%%%%%%%%%%%%%%%%%%%%%%%%%%%%%%%%%%%%%%%%%%%%%%%%%%%%%%%%%%%%%%%%%
\Section{Properties and Filters}\null

\atindex{properties}{|indexit}\atindex{filters}{|indexit}%
\index{methods!selection}%
Certain attributes, like `IsAbelian', are boolean-valued. Such attributes
are known to {\GAP} as *properties*, because their values are stored in a
slightly different way.   A property also  has a  getter, a  setter and a
tester, but  in this case,  the  getter as well  as the  tester returns a
boolean value. Therefore {\GAP}  stores   both values  in the same   way,
namely as bits  in a boolean  list, thereby treating property getters and
all testers (of attributes or properties) uniformly. These boolean-valued
functions  are called  *filters*.  You can imagine  a filter  as a switch
which is  set either  to `true' or  to `false'.  For every  {\GAP} object
there is a boolean list which has reserved a  bit for every filter {\GAP}
knows  about. Strictly speaking, there   is one bit for every  *simple
filter*, and these  simple filters can be  combined with `and' to form
other filters (which are then `true' if and only if all the corresponding
bits are set to `true').
For example, the filter `IsPermGroup and IsSolvableGroup' is made up from
several simple filters.

Since they allow only two values, the bits which represent filters can be
compared very quickly, and the scheme by which {\GAP} chooses the method,
e.g., for a getter or a setter (as we have seen in the previous section),
is mostly based  on the examination of filters,  not on the examination
of other  attribute   values. Details  of   this *method selection*   are
described in chapter~"prg:Method Selection" of ``Programming in GAP''.

We  only present the following  rule  of  thumb here:
Each installed method for an attribute, say `Size',
has a ``required filter'', which is made  up from certain simple filters
which must yield `true' for the argument <obj> for this method to be
applicable.
To execute a call of `Size( <obj> )', {\GAP} selects among all applicable
methods the one whose required filter combines the most simple filters;
the idea behind is that the more an algorithm requires of <obj>,
the more efficient it is expected to be.
For example, if <obj> is a permutation group that is not (known to be)
solvable,
a method  with required filter `IsPermGroup and IsSolvableGroup' is not
applicable, whereas a method with required filter `IsPermGroup' can be
chosen.
On the other hand, if <obj> was  known to be solvable,
the method with required filter `IsPermGroup and IsSolvableGroup'
would be preferred to the one with required filter `IsPermGroup'.

It may happen that a method is applicable for a given argument
but cannot compute the desired value.
In such cases, the method will execute the statement `TryNextMethod();',
\indextt{TryNextMethod}%
and {\GAP} calls the next applicable method.
For example, \cite{Sims90b} describes an algorithm to compute the size
of a solvable permutation group, which can be used also to decide
whether or not a permutation group is solvable.
Suppose that the function `size_solvable' implements this algorithm,
and that is returns the order of the group if it is solvable and
`fail' otherwise.
Then we can install the following method for `Size' with required
filter `IsPermGroup'.
\begintt
function( G )
local  value;
    value := size_solvable( G );
    if value <> fail  then  return value;
                      else  TryNextMethod();  fi;
end;
\endtt
This method can then be tried on every permutation group (whether known
to be  solvable or  not),  and it would  include a  mandatory solvability
test.

If no applicable method  (or no next applicable  method) is found, {\GAP}
stops with an error message of the form 
\begintt
For debugging hints type ?Recovery from NoMethodFound
Error no 1st choice method found for `Size' on 1 arguments at....
\endtt

You would get an error message as above if you asked for `Size( 1 )'. 
The message simply says that there is no method installed for calculating
the size of `1'. Section "ref:Recovery from NoMethodFound-Errors" of
the reference manual contains more information on how to deal with 
these messages.

{\bf Summary.} In this section we have introduced properties as special
attributes, and filters as the general concept behind property getters
and attribute testers.
The values of the filters of an object govern how the object is treated
in the selection of methods for operations.

%%%%%%%%%%%%%%%%%%%%%%%%%%%%%%%%%%%%%%%%%%%%%%%%%%%%%%%%%%%%%%%%%%%%%%%%%
\Section{Immediate and True Methods}\null

\index{methods!immediate}\index{methods!true}%
In the example in Section~"Properties and Filters",
we have mentioned that the operation `Size' has a
method  for solvable permutation  groups that is  so  far superior to the
method for  general permutation groups that  it seems worthwhile to try it
even if nothing  is  known about solvability   of the group of which  the
`Size' is to   be  calculated. There are   other  examples where  certain
methods  are even ``cheaper'' to  execute. For example,  if the size of a
group is known  it is easy to check  whether  it is  odd, and  if so, the
Feit-Thompson  theorem allows us to set  `IsSolvableGroup' to `true' for
this group.   {\GAP} utilizes   this  celebrated  theorem  by  having  an
*immediate  method* for `IsSolvableGroup'  with required filter `HasSize'
which checks parity of the size and either sets `IsSolvableGroup' or does
nothing, i.e.,   calls `TryNextMethod()'.   These immediate  methods  are
executed  automatically for an  object  whenever the   value of a  filter
changes, so solvability of a group will automatically be detected when an
odd size has been calculated for it (and therefore the value of `HasSize'
for that group has changed to `true').

Some methods are  even more immediate,   because they do not  require any
calculation  at all: They  allow a filter to  be set if another filter is
also set. In other words,   they  model a mathematical implication   like
`IsGroup   and   IsCyclic    $\Rightarrow$   IsSolvableGroup'  and   such
implications  can be installed  in {\GAP}  as  *true methods*. To execute
true methods, {\GAP} only needs to do  some bookkeeping with its filters,
therefore true methods are much faster than immediate methods.

How immediate and true methods are installed is described in 
"prg:Immediate Methods" and "prg:Logical Implications"
in ``Programming in GAP''.


%%%%%%%%%%%%%%%%%%%%%%%%%%%%%%%%%%%%%%%%%%%%%%%%%%%%%%%%%%%%%%%%%%%%%%%%%
\Section{Operations and Method Selection}\null

\index{operations}%
The method selection  is not only  used to  select  methods for attribute
getters but also for arbitrary *operations*, which can have more than one
argument.  In this case,   there is a   required filter for each argument
(which must yield `true'  for the corresponding  arguments).

Additionally, a method with at least two arguments may require a certain
relation between the arguments,
which is expressed in terms of the *families* of the arguments.
For example, the methods for  `ConjugateGroup( <grp>, <elm> )'
require that <elm>  lies in the family   of elements from  which <grp> is
made, i.e., that  the family of  <elm> equals the ``elements family''  of
<grp>.

For permutation groups, the situation is quite easy:
all permutations form  one family, `PermutationsFamily',
and each collection of permutations,
for example each permutation group, each coset of a permutation group,
or each dense list of permutations,
lies in `CollectionsFamily( PermutationsFamily )'.

For other kinds of group elements, the situation can be different.
Every call of `FreeGroup' constructs a new family of free group elements.
{\GAP} refuses to compute `One( FreeGroup( 1 ) ) * One( FreeGroup( 1 ) )'
because the two operands of the multiplication lie in different families
and no method is installed for this case.

For further information on family relations,
see "ref:Families" in the reference manual.

\indextt{KnownPropertiesOfObject}\indextt{KnownTruePropertiesOfObject}%
\indextt{KnownAttributesOfObject}%
If you want to know  which properties are already known for an object
<obj>, or which properties are known to be true, you can use the
functions `KnownPropertiesOfObject(<obj>)' resp.
`KnownTruePropertiesOfObject( <obj> )'. This will print a list of names
of properties. These names are also the identifiers of the property
getters, by which you can retrieve the value of the properties (and
confirm that they are really `true'). Analogously, there is the function
`KnownAttributesOfObject' which lists the names of the known attributes,
leaving out the properties.

\indextt{ApplicableMethod}%
Since {\GAP} lets you know what it already  knows about an object, it is
only natural  that   it also  lets   you know what  methods  it considers
applicable for a certain method, and in  what order it  will try them (in
case `TryNextMethod()'  occurs).   `ApplicableMethod( <opr>, [   $arg_1$,
$arg_2$, \dots\ ]  )' returns the  first  applicable method for the  call
`<opr>( $arg_1$, $arg_2$,  \dots\ )'. More  generally, `ApplicableMethod(
<opr>, [ \dots\ ], 0, <nr> )' returns the <nr>th applicable method (i.e.,
the one  that would be  chosen  after  $<nr>-1$ `TryNextMethod's) and  if
`<nr>  = "all"', the sorted list  of  all applicable methods is returned.
For  details,  see "prg:Applicable Methods  and  Method Selection" in
``Programming in GAP''.

\indextt{TraceMethods}%
If you want to see which methods  are chosen for certain operations while
{\GAP}  code is being executed,  you can call the function `TraceMethods'
with a list of these operations as arguments.
\beginexample
gap> TraceMethods( [ Size ] );
gap> g:= Group( (1,2,3), (1,2) );;  Size( g );
#I  Size: method for a permutation group
#I  Setter(Size): system setter
#I  Size: system getter
#I  Size: system getter
6
\endexample
The system getter is called once to fetch  the freshly computed value for
returning  to the user.  The  second  call is  triggered by  an immediate
method. To  find out  by which,  we can trace   the immediate  methods by
saying `TraceImmediateMethods( true )'.
\beginexample
gap> TraceImmediateMethods( true );
gap> g:= Group( (1,2,3), (1,2) );;
#I  immediate: Size
#I  immediate: IsCyclic
#I  immediate: IsCommutative
#I  immediate: IsTrivial
gap> Size( g );
#I  Size: for a permutation group
#I  immediate: IsNonTrivial
#I  immediate: Size
#I  immediate: IsNonTrivial
#I  Setter(Size): system setter
#I  Size: system getter
#I  immediate: IsPerfectGroup
#I  Size: system getter
#I  immediate: IsEmpty
6
gap> TraceImmediateMethods( false );
gap> UntraceMethods( [ Size ] );
\endexample
The last two lines switch off tracing  again. We now  see that the system
getter was called by the immediate method for `IsPerfectGroup'. 
Also the above-mentioned immediate method for `IsSolvableGroup'
was not used because the solvability of `g'  was already found out during
the size calculation
(cf.~the example in Section~"Properties and Filters").

{\bf Summary.}  In this section  and the  last we have  looked some  more
behind the  scenes and seen  that {\GAP} automatically executes immediate
and true  methods  to deduce  information about  objects  that is cheaply
available.  We  have seen how   this  can be  supervised  by tracing  the
methods.

%%%%%%%%%%%%%%%%%%%%%%%%%%%%%%%%%%%%%%%%%%%%%%%%%%%%%%%%%%%%%%%%%%%%%%%%%
%%
%E  opers.tex . . . . . . . . . . . . . . . . . . . . . . . . . ends here

