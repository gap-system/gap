%%%%%%%%%%%%%%%%%%%%%%%%%%%%%%%%%%%%%%%%%%%%%%%%%%%%%%%%%%%%%%%%%%%%%%%%%
%%
%W  sharepkg.tex              GAP documentation             Werner Nickel
%%
%H  @(#)$Id$
%%


%%%%%%%%%%%%%%%%%%%%%%%%%%%%%%%%%%%%%%%%%%%%%%%%%%%%%%%%%%%%%%%%%%%%%%%%%
\Chapter{Share Packages}

A {\GAP} share package  is a part of the  {\GAP} distribution that is not
automatically  available  when {\GAP}  is  started.   A  share package is
activated by the function   RequirePackage().  Share packages reside   in
subdirectories  of  the directory pkg which  itself  is one of the dozend
subdirectories of a typical {\GAP} installation.

A share package extends  the functionality of   {\GAP} as defined by  the
{\GAP} kernel,  the   {\GAP}  library  and  the  various  data libraries.
Typically, a share package includes one or more standalone programs which
perfom tasks that either are not available in  {\GAP} or can be performed
much faster by purpose written programs.  However, this is not always the
case:  some share library packages   are entirely  written in the  {\GAP}
language.

The responsibility  of a share package remains   with the original author
while the   responsibility of the rest of   {\GAP} lies  with  the {\GAP}
developer team.  A share   package undergoes a formal refereeing  process
before it becomes  part of the  {\GAP} distribution.  This process is  in
many  ways similar to  the refereeing process  of a  paper submitted to a
journal.  It assesses the quality and usefulness of the submitted package
and makes sure that it can be started  and runs smoothly.  Share packages
should be submitted  to the chaiman of  the {\GAP} council, Prof. Charles
Wright.

A share package is the way to make software written by (groups of) {\GAP}
users available to the computational community.

% share packages need not be available on all platforms, due to
% incompatibilities between standalones and the OS.

%%%%%%%%%%%%%%%%%%%%%%%%%%%%%%%%%%%%%%%%%%%%%%%%%%%%%%%%%%%%%%%%%%%%%%%%%
\Section{The Structure of a Share Package}

Every share  package  is in a  subdirectory  of  the directory  pkg which
itself is a  subdirectory of a {\GAP}  root directory.  *The name of this
directory is  the name of  the share  package.* It is  this name  that is
passed  as  the only argument  to  RequirePackage() when  the  package is
loaded.  The share package directory must contain a file called 'init.g'.
This is only file of a share library package that is directly read by the
function RequirePackage().  'init.g'  contains the necessary {\GAP}  code
to setup all that the share package needs.

To illustrate this, create  the following directories  in your home area:
'pkg' and  'pkg/example'.  Inside the directory  'example'  create the
file 'init.g' with the single line
\begintt
'Print( "reading the init file of share package example" )'
\endtt
The next bit is a bit tricky  because you have  to find out what the root
directories of {\GAP}  on your  system are.   This  is done by   starting
{\GAP}  and looking  at the  variable  'GAP_ROOT_PATHS'.  This a list  of
directories which {\GAP}  searches upon   start  up for things like   the
{\GAP} library.
\begintt
gap> GAP_ROOT_PATHS;
[ "/gap/4.0/" ]
gap> 
\endtt
Now start {\GAP} with the command 
\beginitems
gap -l \"./;/gap/4.0/\"
\enditems
The  string between the  pair  of double quotes    are the components  of
'GAP_ROOT_PATHS' seperated by semicolons.  We have added at the beginning
the string  './' denoting the  current directory.  This  adds the current
directory to the list of {\GAP} root directories.   Now you can load your
share package:
\beginitems
gap> RequirePackage("example");
\#I  Reading init file of the share package example.
\enditems
This is a very rudimentary share package.  It does not do anything useful
yet, but we have succeeded in loading it via 'RequirePackage()'.


%%%%%%%%%%%%%%%%%%%%%%%%%%%%%%%%%%%%%%%%%%%%%%%%%%%%%%%%%%%%%%%%%%%%%%%%%
\Section{Writing a Share Package}

%%%%%%%%%%%%%%%%%%%%%%%%%%%%%%%%%%%%%%%%%%%%%%%%%%%%%%%%%%%%%%%%%%%%%%%%%
\Section{Standalone Programs in a Share Package}

%%%%%%%%%%%%%%%%%%%%%%%%%%%%%%%%%%%%%%%%%%%%%%%%%%%%%%%%%%%%%%%%%%%%%%%%%
\Section{Relevant Functions}

\>RequirePackage( <name> )

checks if the share package with name <name> has been initialised earlier
in  which   case  nothing  happens.   Otherwise    the share   package is
initialised  by reading the file  'init.g'  from the corresponding  share
package directory.

\>ReadPkg( <name> )

\>DeclarePackageDocumentation( <name> )
\>ReadAsFile( <filename> )
%LastSystemError()
%DirectoriesPackageProgram()

%%%%%%%%%%%%%%%%%%%%%%%%%%%%%%%%%%%%%%%%%%%%%%%%%%%%%%%%%%%%%%%%%%%%%%%%%
\Section{Writing Documentation}

%%%%%%%%%%%%%%%%%%%%%%%%%%%%%%%%%%%%%%%%%%%%%%%%%%%%%%%%%%%%%%%%%%%%%%%%%
%%
%E  Emacs . . . . . . . . . . . . . . . . . . . . . local emacs variables
%%
%%  Local Variables:
%%  fill-column:    73
%%  End:
%%
