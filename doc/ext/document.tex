%%%%%%%%%%%%%%%%%%%%%%%%%%%%%%%%%%%%%%%%%%%%%%%%%%%%%%%%%%%%%%%%%%%%%%%%%
%%
%W  document.tex              GAP documentation              Frank Celler
%%
%H  @(#)$Id$
%%
%Y  Copyright 1997,  Lehrstuhl D fuer Mathematik,  RWTH Aachen,   Germany
%%
%%  This file describes the manual format.
%%

%%%%%%%%%%%%%%%%%%%%%%%%%%%%%%%%%%%%%%%%%%%%%%%%%%%%%%%%%%%%%%%%%%%%%%%%%
\Chapter{Manual Format}

This  chapter describes  the  manual format  used to generate  the {\GAP}
manual and the on-line help. The {\TeX} source of the manual is also used
to generate the  on-line help. Therefore some  restrictions apply to the
{\TeX} macros one should use. These restriction are described in sections
"TeX Macros" and "Examples, Lists, and Verbatim".

The first sections "The main file"  and "Chapters and Sections"  describe
the  general  layout of the  files  in case  you need to   write your own
package documentation.

%%%%%%%%%%%%%%%%%%%%%%%%%%%%%%%%%%%%%%%%%%%%%%%%%%%%%%%%%%%%%%%%%%%%%%%%%
\Section{The Main File}

The main {\TeX} file is called ``manual.tex''.
This file should contain only the following commands:

%\begintt
\){\\input ../gapmacro.tex}
\){\\BeginningOfBook\{<name-of-book>\}}
\){\ \ \\UseReferences\{<book1>\}}
\){\ \ ...}
\){\ \ \\UseReferences\{<bookn>\}}
\){\ \ \\TitlePage\{<title>\}}
\){\ \ \\Colophon\{<text>\}}
\){\ \ \\TableOfContents}
\){\ \ \\FrontMatter}
\){\ \ \ \ \\immediate\\write\\citeout\{\\bs bibdata\{<mybibliography>\}\}}
\){\ \ \ \ \\Input\{<file1>\}}
\){\ \ \ \ ...}
\){\ \ \ \ \\Input\{<filen>\}}
\){\ \ \\Chapters}
\){\ \ \ \ \\Input\{<file1>\}}
\){\ \ \ \ ...}
\){\ \ \ \ \\Input\{<filen>\}}
\){\ \ \\Appendices}
\){\ \ \ \ \\Input\{<file1>\}}
\){\ \ \ \ ...}
\){\ \ \ \ \\Input\{<filen>\}}
\){\ \ \ \ \\Bibliography}
\){\ \ \ \ \\Index}
\){\\EndOfBook}
%\endtt

The first  line inputs the  file ``gapmacro.tex''.  If  you are writing a
share package  either copy this file or use a  relative path.  The former
method will always work but requires you to keep the file consistent with
the  system while  the latter  forces users to  change the ``manual.tex''
file  if they are installing  a package in  a private location.  See also
Section "ref:GAP Root Directory" in the Reference Manual.

`\\BeginningOfBook' starts   the   book, <name-of-book>   is  used    for
cross-references,  see "Labels  and References".   If  you are  writing a
share package use the name of your package here.

If   your manual   consists   of more    than    one book  the    command
`\\UseReferences' can  be used to  load the labels  of the other books in
case  cross-references  occur.  <booki>  is  the  path  of the  directory
containing  the book  whose  references you want  to   load.  If you  are
writing a share package and you need to reference the main {\GAP} manual,
use  `\\UseReferences' for each book  you want to reference.  However, as
said  above  this requires  changes to   the ``manual.tex'' file   if the
package is not installed in the standard location.

*Example*

If your ``manual.tex'' file lives in ``pkg/qwer/doc'' and you want to use
references to the tutorial use

\begintt
\UseReferences{../../../doc/tut}
\endtt

`\\TitlePage' produces a page containing the `title'.

`\\Colophon' produces a page following the title that can be used for more
explicit author information, acknowledgements, dedications or whatsoever.

`\\TableOfContents' produces a table of contents.

`\\FrontMatter' starts the front matter chapters such as
a copyright notice or a preface.

The line
\begintt
\immediate\write\citeout{\bs bibdata{<mybibliography>}}
\endtt
is for users of Bib{\TeX}. It will use the file `<mybibliography>.bib' to
fetch bibliography information.


`\\Chapters'  starts the chapters  of the manual,  which are included via
`\\Input'.  For the chapter format, see Section~"Chapters and Sections".

`\\Appendices'  starts the appendices.
%`\\Answers'  produces an answers chapter, see  "Excersises  and  Answers".
`\\Bibliography'  produces a bibliography, and `\\Index' an index.

Finally `\\EndOfBook' closes the book.

*Example*

Assume you have a share package ``qwert'' with two chapters ``Qwert'' and
``Extending  Qwert'', a copyright  notice,  a preface, no exercises, then
your ``manual.tex'' would basically look like:

\begintt
\input gapmacro.tex
\BeginningOfBook{qwert}
  \TitlePage{
    \centerline{\titlefont The Share Package ``qwert''}
    \centerline{\secfont by}
    \centerline{\titlefont Q. Mustermensch}
  }
  \TableOfContents
  \FrontMatter
    \Input{copyright}
    \Input{preface}
  \Chapters
    \Input{qwert}
    \Input{extend}
  \Appendices
    \Index
\EndOfBook
\endtt

%%%%%%%%%%%%%%%%%%%%%%%%%%%%%%%%%%%%%%%%%%%%%%%%%%%%%%%%%%%%%%%%%%%%%%%%%
\Section{Chapters and Sections}

The command `\\Chapter\{<chaptername>\}' starts a new chapter named
<chaptername>, a chapter begins with an introduction to the chapter and
is followed be sections created with the `\\Section\{<secname>\}'
command. The strings <chaptername> and <secname> are automatically
available as references.

There must be *no further commands* on the same line as the `\\Chapter'
or `\\Section' line,
and there *must* be an empty line after a `\\Chapter' or `\\Section'
command.
This means that `\\index' commands referring to the chapter or section
can be placed only after this empty line.

The contents of each chapter must be in its own `.tex' file.

To use the HTML converter, each `\\Section' line must be preceded by a line
full of percentage signs.
However the converter will stop converting a section whenever it his such a
line, therefore do not add lines with 16 or more % signs which are *not*
just before a `\\Section' command.



%%%%%%%%%%%%%%%%%%%%%%%%%%%%%%%%%%%%%%%%%%%%%%%%%%%%%%%%%%%%%%%%%%%%%%%%%
\Section{TeX Macros}

As the manual pages are  also used as on-line help,
and are automatically converted to HTML,
the use of special {\TeX} commands should be avoided.
The following macros can  be used to  structure  the text, the  mentioned
fonts are used  when printing the manual,  however  the on-line help  and
HTML are free to use other fonts or even colour.

\beginitems

`{`text'}' &
    sets the `text' in typewriter style.
    This is typically used to denote {\GAP} keywords such as `for' and
    `false' or variables that are not arguments to a function,
    see also `\<text>'.
    Use `\\\<' to get a ``less than'' sign.

`{``text''}' &
    encloses the text in doublequotes.  In particular this does *not* set
    `text'    in typewriter   style,   use  `{`\{`text'\}'}' to   produce
    `{`text'}'.  Doublequotes are mainly used to mark a phrase which will
    be defined later or is used in an uncommon way.

`\\pif' &
    sets a single apostrophe.

`\<text>' &
    sets the text in italics.  This can also be used inside `\$...\$' and
    `{`...'}'.
    Use `\\\<' to get a ``less than'' sign.
    `\<text>' is used to denote a variable which is an argument of a
    function, a typical application is the description of a function:
\begintt
\>Group( <U> )
The function `Group' constructs a group isomorphic to <U>.
\endtt

`*text*' &
    sets the text in emphasized style.

`\$a.b\$' &
    Inside math mode, you can use `.' instead of `\\cdot'.
    Use `\\.' for a full stop inside math mode.
    For example, `\$a.b\$' produces $a.b$
    while `\$a\\.b\$' produces $a\.b$.

`\\cite\{...\}' &
    produces  a     reference     to    a    bibliography   entry    (the
    `\\cite[...]\{...\}' option of La{\TeX} is *not* supported).

`"ref"' &
    produces a reference to a label.
    Labels are generated by the commands `\\Chapter', `\\Section'
    (see~"Labels and References").

`\\index\{...\}' &
    defines an index entry.  Index entries are  also used for the section
    index file `manual.six' used by the on-line help.
    An exclamation mark (`!') may be used to indicate index subentries.

`\\indextt\{...\}' &
    is the same as `\\index\{...\}', except that the index entry
    is printed in typewriter style.

`\{\\GAP\}' &
    typesets {\GAP}.

`\\>' &
    produces a subsection for a function.
    This macro uses the brackets of the function to parse the arguments
    and therefore requires the function to use brackets and the arguments
    to have none.
    It also creates automatically an index entry.
    The line following the `\\>' entry must either contain another `\\>'
    entry (in which case the further entries are assumed to be variants
    and do not start a new subsection) or must be empty.
    The description text will follow this empty line.
    Here is an example how to use `\\>'; the index entry is `Size'.
\begintt
\>Size( <obj> )
\endtt

`\\>{`<command>'}\{<label>\}' &
    Works as `\\>' but will not use bracket matching but simply display
    <command> as a header.
    It will use <label> as an index entry.
\begintt
\>`<a> + <b>'{summation}
\>`Size(<obj>)'{size}
\endtt
    In the first of the above examples, `\\>' cannot be used because
    no brackets occur.
    In the second example, the difference to the use of `\\>' is
    that the index entry will be typeset as ``size'' instead of ``Size''.

`\\>{`<command>'}\{<label>\}@\{<text>\}' &
    Works as `\\>' but will not use bracket matching but simply display
    <command> as a header.
    It will use <label> for sorting the index entry which will be printed
    as <text>.
    Here are two examples.
\begintt
\>`Size(<obj>)'{size}@{`Size'}
\>`Size(GL(<n>,<q>))'{size!gl}@{`Size(GL(<n>,<q>))'}
\endtt

`\\)\{\\fmark ...\}' &
    is like `\\>' except that it produces no label and index entry.

`\\URL\{<url>\}' &
Prints the WWW URL <url>. In the HTML version this will be a HREF link.

`\\Mailto\{<email>\}' &
Prints the email address <email>. In the HTML version this will be a
`mailto' link.
\enditems

%%%%%%%%%%%%%%%%%%%%%%%%%%%%%%%%%%%%%%%%%%%%%%%%%%%%%%%%%%%%%%%%%%%%%%%%%
\Section{Examples, Lists, and Verbatim}

% produce itemized texts with 3pc hanging indentation
In order  to   produce  a  list   of  items with   descriptions  use  the
`\\beginitems', `\\enditems' environment.

For example, the   following  list describes   `base', `knownBase',  and
`reduced'.
The different item/description pairs must be separated by blank lines.

\begintt
\beginitems
`base' &
    must be a  list of points ...

`knownBase' &
    If a base for <G> is known in advance ...

`reduced' (default `true') &
    If this is `true' the resulting stabilizer chain will be ...
\enditems
\endtt

This will be printed as
\beginitems
`base' &
    must be a  list of points ...

`knownBase' &
    If a base for <G> is known in advance ...

`reduced' (default `true') &
    If this is `true' the resulting stabilizer chain will be ...
\enditems


In order to produce a list in a more compact format,
use the `\\beginlist', `\\endlist' environment.

An example is the following list.

\begintt
\beginlist
\item{(a)}
    first entry
\item{(b)}
    second entry
\item{(c)}
    third entry
\endlist
\endtt

It is printed as follows.
\beginlist
\item{(a)}
    first entry
\item{(b)}
    second entry
\item{(c)}
    third entry
\endlist

% verbatim text in typewriter style
Example {\GAP} sessions are typeset in typewriter style
using the `\\beginexample', `\\endexample' environment.

\begintt
\beginexample
gap> 1+2;
3
\endexample
\endtt

typesets the example
\beginexample
gap> 1+2;
3
\endexample

Non-{\GAP} examples are typeset in typewriter style
using the `\\begintt', `\\endtt' environment.

The manual style will automatically indent examples. It also will break
examples which become too long to fit on one page. If you want to encourage
breaks at specific points in an example, end the example with `\\endexample'
and immediately start a new example environment with `\\beginexample' on
the next line.

%%%%%%%%%%%%%%%%%%%%%%%%%%%%%%%%%%%%%%%%%%%%%%%%%%%%%%%%%%%%%%%%%%%%%%%%%
\Section{Testing the Examples}

For purposes of automatically checking the manual,
the {\GAP} examples in one chapter (the text between `\\beginexample'
and `\\endexample') should produce the same output, up to line breaks
and whitespace, whenever they are run in the same order
immediately after starting {\GAP}
(this will ensure that the global random number generator is initialized
to the *same* values).
For more details,
see the last paragraph of~"tut:Starting and Leaving GAP" in the Tutorial.

To permit this automatic running,
examples that shall produce error messages should be put between
`\\begintt' and `\\endtt'
such that they will not be seen by this automatic test.

The automatic test also requires that examples are not indented
in the files;
in the printed manual,
the lines between `\\beginexample' and `\\endexample'
and the lines between `\\begintt' and `\\endtt' are automatically
indented.

% mention how this test can be executed!


%%%%%%%%%%%%%%%%%%%%%%%%%%%%%%%%%%%%%%%%%%%%%%%%%%%%%%%%%%%%%%%%%%%%%%%%%
\Section{Using TeX Macros}

As  described in~"TeX Macros", the use of  macros should be restricted to
the ones given in the previous sections.  However, in rare situations one
might be forced to use other {\TeX} macros,
for example  in order  to  typeset a lattice.
In this case you should provide an alternative for the on-line help.
This can be done by putting in guiding commands as {\TeX} comments:

\begintt
%display{tex}
TeX version (only used by TeX manual)
%display{html}
%HTML version (only used by HTML manual)
%display{text}
%Text version (only used by the built-in manual browser)
%enddisplay
\endtt

However, this use of special macros should be limited to a minimum.

%%%%%%%%%%%%%%%%%%%%%%%%%%%%%%%%%%%%%%%%%%%%%%%%%%%%%%%%%%%%%%%%%%%%%%%%%
\Section{Umlauts}
To produce Umlauts use `\\accent127' and not the shorthand
\beginverbatim
\"
\endverbatim
Otherwise the HTML converter will not translate it properly.

%%%%%%%%%%%%%%%%%%%%%%%%%%%%%%%%%%%%%%%%%%%%%%%%%%%%%%%%%%%%%%%%%%%%%%%%%
\Section{Producing a Manual}

To produce a manual you will need the following files:

\beginitems
`manual.tex'&
    contains the body of the manual
    (as described in Section~"The Main File")
    and `\\Input' commands for other files.

`gapmacro.tex'&
    contains the macros for the manual.
    It must be input by an `\\Input' statement in `manual.tex'.
    You can either use the version in the `doc' directory of {\GAP}
    (use a relative path then) or make a copy.

`manual.mst'&
    is used by `makeindex' to fetch index information.
    It must reside in your manual directory.

`GAPDOCPATH/manualindex'&
    is used to call `makeindex'.
    `GAPDOCPATH' is the path of the `doc' directory of your {\GAP}
    distribution.
\enditems

For bibliography information you will need a file `manual.bbl'. If you
intend to create it with Bib{\TeX}, you will need to indicate the
appropriate `.bib' file (as described in section "The main file"). Then
after running {\TeX} once over the manual, run Bib{\TeX} to create the
`manual.bbl' file.

Assuming that all necessary files are there,
on a Unix system the following calls will then produce a file `manual.dvi'
as well as a file `manual.six' which is used by the GAP help functions:

Go to the directory holding the manual. Call
\begintt
tex manual
\endtt
to produce bibliography information. Unless you provide a `manual.bbl' file
which is not produced by Bib{\TeX}, call
\begintt
bibtex manual
\endtt
to produce the `manual.bbl' file. Then run {\TeX} twice over the manual to
fill all references and produce a stable table of contents:
\begintt
tex manual
tex manual
\endtt
If you have sections which are named like commands, you may get messages
about redefined labels. At this point you can ignore these.

Now it is time to produce the index. Call
\begintt
GAPDOCPATH/manualindex manual
\endtt
to run `makeindex'. This produces a file `manual.ind'. Finally, once again
run
\begintt
tex manual
\endtt
to incorporate the index. The manual is ready.

%%%%%%%%%%%%%%%%%%%%%%%%%%%%%%%%%%%%%%%%%%%%%%%%%%%%%%%%%%%%%%%%%%%%%%%%%
%%
%E

