%%%%%%%%%%%%%%%%%%%%%%%%%%%%%%%%%%%%%%%%%%%%%%%%%%%%%%%%%%%%%%%%%%%%%%%%%%%%%
\Chapter{Manual Format}

This chapter  describes  the manual  format used  to produces the  {\GAP}
manual and the on-line help.   The {\TeX}  source of  the manual is  also
used to generated the on-line help.   Therefore some restriction apply to
the {\TeX}  macros  one can   use.   These restriction are  described  in
sections "TeX Macros" and "Examples, Lists, and Verbatim".

The first sections "Manual  Files"  and "Chapters and Sections"  describe
the  general  layout of the  files  in case  you need to   write your own
package documentation.

%%%%%%%%%%%%%%%%%%%%%%%%%%%%%%%%%%%%%%%%%%%%%%%%%%%%%%%%%%%%%%%%%%%%%%%%%%%%%
\Section{Manual Files}

`\\TitlePage''
`\\TableOfContents'
`\\FrontMatters'
`\\Chapters'
  `\\Input'
`\\Appendices'
  `\\Index'


%%%%%%%%%%%%%%%%%%%%%%%%%%%%%%%%%%%%%%%%%%%%%%%%%%%%%%%%%%%%%%%%%%%%%%%%%%%%%
\Section{Chapters and Sections}

The  command `\\Chapter\{<chaptername>\}'  starts  a   new chapter  named
<chaptername>, a chapter beginns with an  introduction to the chapter and
is followed  be   sections  created  with  the   `\\Section\{<secname>\}'
command.   The strings  <chaptername>  and   <secname> are  automatically
available as references.


\\Section{<secname>}\null

%%%%%%%%%%%%%%%%%%%%%%%%%%%%%%%%%%%%%%%%%%%%%%%%%%%%%%%%%%%%%%%%%%%%%%%%%%%%%
%  \Chapter{title}\par \Section{title}\par
      make chapter or section title. Automatically generates table of
%      contents, a label and an index entry. \Section{...}\null
      inhibits labels and indexing.
%  \FrontMatter, \Chapters, \Appendices  parts of the book
%  \Bibliography, \TableOfContents       make these chapters automatically
%  \Index                                make index without chapter head

%  \>function( arguments )!{ index subentry }
%  \>`a binop b'{binary operation}!{ index subentry }
      make a  heading for a subsection   explaining a function  or a binary
      operation. This automatically generates   a label and an  index entry
      (with optional subentry).
%  \){\fmark ...}
      the same without label and index entry. \fmark is the triangle.

  These last three probably won't occur in the tutorial.


%%%%%%%%%%%%%%%%%%%%%%%%%%%%%%%%%%%%%%%%%%%%%%%%%%%%%%%%%%%%%%%%%%%%%%%%%%%%%
\Section{TeX Macros}

As the manual pages are  also used as  on-line help and are automatically
converted  to HTML the  use of  special {\TeX}  commands should be avoid.
The following  macros can be  used to  structure the  text, the mentioned
fonts are used when printed the manual, however the on-line help and HTML
are free to use other fonts or even colour.

\beginitems

`{`text'}' &
    sets the `text' in typewrite style. This is typically used to denoted
    {\GAP}  commands like `for' or  variables like `false'  which are not
    arguments to a function, see also `\<text>'.

`{``text''}' &
    encloses the text in doublequotes.  In particular this does *not* set
    `text'    in typewriter   style,   use  `{`\{`text'\}'}' to   produce
    `{`text'}'.  Doublequotes are mainly used to mark a phrase which will
    be defined later or is used in an uncommon way.
    
`\<text>' &
    sets the text in italics.  This can also be used inside `\$...\$' and
    `{`...'}'. Use `\<' to  get a less than sign.   `\<text>' is  used to
    denote  a  variable which is  a  argument of a  function, a typically
    application is the description of a function.
\begintt
        Group( <U> )
        The function `Group' constructs a group $G$ isomorphic to <U>.
\endtt

`*text*' &
    sets the text in emphasized style.

`\$a.b\$' &
    inside math mode, you can use `.'  instead of `\\cdot'. Use `\\.' for
    a  full  stop in `\$...\$'.   For   example, `\$a.b\$' produces $a.b$
    while `\$a\\.b\$' produces $a\.b$.

`\\cite\{...\}' &
    produces  a     reference     to    a    bibliography   entry    (the
    `\\cite[...]\{...\}' option of LaTeX is not supported).

`"ref"' &
    produces  a reference  to a  label.  References  are generated by the
    `\\Chapter', `\\Section', *@WHAT ELSE@* commands.

`\\index\{...\}' &
    defines an index entry.  Index entries are  also used for the section
    index `tutorial.six' used by the on-line help.

`\\GAP' &
    typesets {\GAP}.

\enditems

%%%%%%%%%%%%%%%%%%%%%%%%%%%%%%%%%%%%%%%%%%%%%%%%%%%%%%%%%%%%%%%%%%%%%%%%%%%%%
\Section{Examples, Lists, and Verbatim}

In order  to   produces a  list   of  items with   descriptions  use  the
`\\beginitems', `\\enditems' environment.

For examples, the   following  list describes   `base', `knownBase',  and
`reduced'.  The different item/description pairs are separated by a blank
line.

\begintt
  \beginitems
    `base' &
        must be a  list of points ...

    `knownBase' &
        If a base for <G> is known in advance ...

    `reduced' (default `true') &
        If this is `true' the resulting stabilizer chain will be ...
  \enditems
\endtt

typesets the list:

\beginitems
  `base' &
      must be a  list of points ...

  `knownBase' &
      If a base for <G> is known in advance ...

  `reduced' (default `true') &
      If this is `true' the resulting stabilizer chain will be ...
\enditems

Example   {\GAP}  sessions  are    typeset  using the   `\\beginexample',
`\\endexample' environment.   They  should contain   the   exact  copy of
{\GAP} session using a line width of 73 and  indented by 4 blanks because
the   manual  checker  uses    the  test  between   `\\beginexample'  and
`\\endexample' to generate a test file.

\begintt
  \beginexample
      gap> 1+2;
      3
  \endexample
\endtt

typesets the example

\beginexample
    gap> 1+2;
    3
\endexample

Other non-{\GAP} examples are typeset using the `\\begintt', `\\endtt'
environment.

%%%%%%%%%%%%%%%%%%%%%%%%%%%%%%%%%%%%%%%%%%%%%%%%%%%%%%%%%%%%%%%%%%%%%%%%%%%%%
\Section{Using TeX macros}

As  described in "TeX Macros"  the use of  macros should be restricted to
the ones given in the previous sections.  However, in rare situations one
might be forced  use  other {\TeX}, for  example  in order  to  typeset a
lattice.  In this case you should provide  an alternative for the on-line
help.

*@WHAT IS EXACT SYNTAX?@*

