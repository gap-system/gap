%%%%%%%%%%%%%%%%%%%%%%%%%%%%%%%%%%%%%%%%%%%%%%%%%%%%%%%%%%%%%%%%%%%%%%%%%%%%
%
%A about.tex           GAP documentation
%
%A @(#)$Id$
%
%Y Copyright 1990-1992, Lehrstuhl D fuer Mathematik, RWTH Aachen, Germany
%
\Chapter{About the GAP Reference Manual}

This is one of four parts of the {\GAP} documentation,
the others being the *{\GAP} Tutorial*, a beginner's introduction to {\GAP},
*Programming in {\GAP}* and *Extending {\GAP}*,
which provide information for those who want to write their own
{\GAP} extensions.

This manual, the *{\GAP} reference manual* contains the official definitions
of {\GAP} functions. It should give all information to someone who wants to
use {\GAP} as it is. It is not intended to be read cover-to-cover.

This manual is divided into chapters.
Each chapter is divided into sections
and, within each section, important definitions are numbered.
References are therefore triples.

Chapter~"The Help System" describes the *help system*,
which provides online access to the information of the manual.
Chapter~"Running GAP" gives technical advice for *running* {\GAP}.
Chapter~"The Programming Language" introduces the {\GAP} language,
while the next chapters deal with the *environment*
provided by {\GAP} for the user.
These are followed by the main bulk of chapters
which is devoted to various mathematical structures that {\GAP} can handle.

Pages are numbered consecutively in each of the four manuals.

%%%%%%%%%%%%%%%%%%%%%%%%%%%%%%%%%%%%%%%%%%%%%%%%%%%%%%%%%%%%%%%%%%%%%%%%
\Section{Manual Conventions}

The printed manual uses different text styles for several purposes.
Note that the online help may use other symbols to express the meanings
listed below, see~"Format of Sections".

\){*text*}

Text printed in boldface is used to emphasize single words or phrases.

\){<text>}

Text printed in italics is used for arguments in the descriptions
of functions and for other place holders. It means that you should not
actually enter this text into {\GAP} but replace it by  appropriate
text depending on what you want to do. For example when we write that
you should enter `?<section>' to see the section with the name <section>,
<section> serves as a place holder, indicating that you can enter the
name of the section that you want to see at this place.

\){text}

Text printed in a monospaced (all characters have the same width)
typewriter font is used for names of variables and functions
and other text that you may actually enter into your computer
and see on your screen.  Such text may contain
place holders printed in italics as described above.  For example
when the information for `IsPrime' says that the form of the call is
`IsPrime( <n> )' this means that you should actually
enter the strings ``IsPrime('' and ``)'', without the quotes,
but replace the `<n>' with the number (or expression)
that you want to test.

\)Oper( <arg1>, <arg2>[, <opt>] ) F

starts a subsection on the command `Oper' that takes two arguments <arg1>
and <arg2> and an optional third argument <opt>.
As in the above example, the letter `F' at the end of a line that starts
with a little black triangle in the left margin indicates that the command
is a simple function.
Other possible letters at the end of such a line are
`A', `P', `O', `C', `R', and `V';
they indicate ``Attribute'', ``Property'', ``Operation'', ``Category'',
``Representation'' (see Chapter~"Types of Objects"), or ``Variable'',
respectively.

In the printed manual, *mathematical formulas* are typeset in italics
(actually math italics), and subscripts and superscripts are actually
lowered and raised.

Longer *examples* are usually paragraphs of their own.
Everything on the lines with the prompts `gap>' and `>', except
the prompts themselves of course, is the input you have to type;
everything else is {\GAP}'s response. In the printed manual,
examples are printed in a monospaced typewriter font.

%%%%%%%%%%%%%%%%%%%%%%%%%%%%%%%%%%%%%%%%%%%%%%%%%%%%%%%%%%%%%%%%%%%%%%%%
\Section{Credit}

The manual tries to give credit to designers and implementors of major parts
of {\GAP}. For many parts of  the GAP code it is  impossible to give
detailed credit, because over the time of its  development many persons have
contributed from first ideas, even in prerunners of GAP such as CAS or
SOGOS, via first  implementations, improvements, and even  total
reimplementations.  The documentation of the code gives further details, but
again, it suffers from the same problem.  We  have attempted to give
attributions with the different chapters of   the manual where this seemed
to be possible, but  we  apologise for all (unavoidable) shortcomings of
this attempt.

%%%%%%%%%%%%%%%%%%%%%%%%%%%%%%%%%%%%%%%%%%%%%%%%%%%%%%%%%%%%%%%%%%%%%%%%%
%%
%E

