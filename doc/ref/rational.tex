\Chapter{Rationals}

The *rationals*  form  a very important field.  On the one hand it is the
quotient field of the integers (see "Integers").  On the other hand it is
the prime field of the fields of characteristic zero  (see "Number Fields").

The former comment suggests the representation actually used.  A rational
is  represented   as   a   pair   of  integers,  called  *numerator*  and
*denominator*.   NumeratorRat  and  denominator are  *reduced*, i.e.,  their
greatest common divisor is 1.  If  the denominator  is 1, the rational is
in fact an integer and is represented as such.   The numerator holds  the
sign of the rational, thus the denominator is always positive.

Because the underlying integer arithmetic can compute with arbitrary size
integers, the  rational arithmetic is  always  exact, even  for rationals
whose numerators and denominators have thousands of digits.

\beginexample
gap> 2/3;
2/3
gap> 66/123;
22/41    # numerator and denominator are made relatively prime
gap> 17/-13;
-17/13    # the numerator carries the sign
gap> 121/11;
11    # rationals with denominator 1 (after cancelling) are integers
\endexample

The usual arithmetic operators `+',`-',`*',`/' can be applied to rational
numbers, `=' tests equality and the comparison operators implement the usual
ordering.

The {\GAP} object `Rationals' is the field domain  of all rationals.  All
set  theoretic  functions  and all field function are applicable to this
domain.

%%%%%%%%%%%%%%%%%%%%%%%%%%%%%%%%%%%%%%%%%%%%%%%%%%%%%%%%%%%%%%%%%%%%%%%%
\Section{IsRat}
\index{test!for a rational}

\>IsRat( <obj> ) C

`IsRat' returns `true' if <obj>,  which can be  an arbitrary object, is a
rational and `false' otherwise.   Integers are rationals with denominator
1,   thus `IsRat' returns  `true' for  integers.   `IsRat' will signal an
error if <obj> is an unbound variable or a procedure call.

\beginexample
gap> IsRat( 2/3 );
true
gap> IsRat( 17/-13 );
true
gap> IsRat( 11 );
true
gap> IsRat( IsRat );
false    # `IsRat' is a function, not a rational 
\endexample

%%%%%%%%%%%%%%%%%%%%%%%%%%%%%%%%%%%%%%%%%%%%%%%%%%%%%%%%%%%%%%%%%%%%%%%%
\Section{NumeratorRat}
\index{numerator!of a rational}

\>NumeratorRat( <rat> ) F

`NumeratorRat'  returns  the numerator of the rational <rat>.   Because  the
numerator holds the sign of the rational it may be any integer.  Integers
are rationals  with  denominator  1, thus  `NumeratorRat'  is  the  identity
function for integers.

\beginexample
gap> NumeratorRat( 2/3 );
2
gap> NumeratorRat( 66/123 );
22    # numerator and denominator are made relatively prime
gap> NumeratorRat( 17/-13 );
-17    # the numerator holds the sign of the rational
gap> NumeratorRat( 11 );
11    # integers are rationals with denominator 1 
\endexample

`DenominatorRat' (see "DenominatorRat") is the counterpart to `NumeratorRat'.

%%%%%%%%%%%%%%%%%%%%%%%%%%%%%%%%%%%%%%%%%%%%%%%%%%%%%%%%%%%%%%%%%%%%%%%%
\Section{DenominatorRat}
\index{denominator!of a rational}

\>DenominatorRat( <rat> ) F

`DenominatorRat' returns the denominator of the rational <rat>.  Because the
numerator  holds the  sign of  the rational  the denominator  is always a
positive integer.   Integers are rationals  with  the denominator 1, thus
`DenominatorRat' returns 1 for integers.

\beginexample
gap> DenominatorRat( 2/3 );
3
gap> DenominatorRat( 66/123 );
41    # numerator and denominator are made relatively prime
gap> DenominatorRat( 17/-13 );
13    # the denominator holds the sign of the rational
gap> DenominatorRat( 11 );
1    # integers are rationals with denominator 1 
\endexample

`NumeratorRat' (see "NumeratorRat") is the counterpart to `DenominatorRat'.
