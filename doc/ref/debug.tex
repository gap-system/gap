%%%%%%%%%%%%%%%%%%%%%%%%%%%%%%%%%%%%%%%%%%%%%%%%%%%%%%%%%%%%%%%%%%%%%%%%%
%%
%W  debug.tex                 GAP manual                    Thomas Breuer
%W                                                       Alexander Hulpke
%W                                                       Martin Schoenert
%%
%H  @(#)$Id$
%%

%%%%%%%%%%%%%%%%%%%%%%%%%%%%%%%%%%%%%%%%%%%%%%%%%%%%%%%%%%%%%%%%%%%%%%%%%
\Chapter{Debugging and Profiling Facilities}

This chapter describes some functions that are useful mainly for
debugging and profiling purposes.

The sections~"ApplicableMethod" and~"Tracing Methods" show how to get
information about the methods chosen by the method selection mechanism
(see~"prg:Method Selection").

The final sections describe functions for collecting statistics about
computations (see "Runtime", "Profiling").


%%%%%%%%%%%%%%%%%%%%%%%%%%%%%%%%%%%%%%%%%%%%%%%%%%%%%%%%%%%%%%%%%%%%%%%%%
\Section{ApplicableMethod}

\>ApplicableMethod( <opr>, <args> )
\)ApplicableMethod( <opr>, <args>, <printlevel>, <nr> )
\)ApplicableMethod( <opr>, <args>, <printlevel>, "all" )

In the first form, `ApplicableMethod' returns the method of highest rank
that is applicable for the operation <opr> with the arguments in the list
<args>.
If no method is applicable then `fail' is returned.

In the second form, if <nr> is a positive integer then `ApplicableMethod'
returns the <nr>-th applicable method for the operation <opr> with the
arguments in the list <args>,
where the methods are ordered according to descending rank.
If less than <nr> methods are applicable then `fail' is returned.

If the fourth argument is the string `"all"' then `ApplicableMethod'
returns a list of all aplicable methods for <opr> with arguments <args>,
ordered according to descending rank.

Depending on the integer value <printlevel>,
additional information is printed.
Admissible values and their meaning are as follows.

\beginlist
\item{0}
    no information,

\item{1}
    information about the applicable method,

\item{2}
    also information about the not applicable methods of higher rank,

\item{3}
    also for each not applicable method the first reason why it is not
    applicable,

\item{4}
    also for each not applicable method all reasons why it is not
    applicable.
\endlist

When a method returned by `ApplicableMethod' is called then
it returns either the desired result or the string `TRY_NEXT_METHOD',
which corresponds to a call to `TryNextMethod' in the method
and means that the method selection would call the next applicable
method.


%%%%%%%%%%%%%%%%%%%%%%%%%%%%%%%%%%%%%%%%%%%%%%%%%%%%%%%%%%%%%%%%%%%%%%%%%
\Section{Tracing Methods}

\>TraceMethods( <oprs> )
\)UntraceMethods( <oprs> )

After the call of `TraceMethods' with a list <oprs> of operations,
whenever a method of one of the operations in <oprs> is called
the information string used in the installation of the method is printed.

This can be switched off for a list <oprs> of operations by calling
`UntraceMethods( <oprs> )'.

\beginexample
    gap> TraceMethods( [ Size ] );
    gap> g:= Group( (1,2,3), (1,2) );;
    gap> Size( g );
    #I  Size: method for a permutation group
    #I  Setter(Size): system setter
    #I  Size: system getter
    #I  Size: system getter
    6
    gap> UntraceMethods( [ Size ] );
\endexample

Also the immediate methods can be traced, but one can switch on or off
the printing only for the whole set of all operations.
To do this,

\>TraceImmediateMethods( <flag> )

is used, where `<flag> = true' switches on and `<flag> = false'
switches off.

\beginexample
    gap> TraceImmediateMethods( true );
    gap> g:= Group( (1,2,3), (1,2) );;
    #I  immediate: IsFinitelyGeneratedGroup
    gap> Size( g );
    #I  Size: method for a permutation group
    #I  immediate: IsFinitelyGeneratedGroup
    #I  immediate: IsCyclic
    #I  immediate: IsFinitelyGeneratedGroup
    #I  Setter(Size): system setter
    #I  Size: system getter
    #I  immediate: IsPerfectGroup
    #I  Size: system getter
    #I  immediate: IsEmpty
    6
    gap> TraceImmediateMethods( false );
    gap> UntraceMethods( [ Size ] );
\endexample

This example gives an explanation for the two calls of the
``system getter'' for `Size'.
Namely, there are immediate methods that access the known size
of the group.
Note that the group `g' was known to be finitely generated already
before the size was computed,
the calls of the immediate method for `IsFinitelyGeneratedGroup'
after the call of `Size' have other arguments than `g'.


%%%%%%%%%%%%%%%%%%%%%%%%%%%%%%%%%%%%%%%%%%%%%%%%%%%%%%%%%%%%%%%%%%%%%%%%%
\Section{Runtime}

\>Runtime(  )

`Runtime' returns the time spent by {\GAP} in milliseconds as an integer.
This is usually the cpu time, i.e., not the wall clock time.
Also time spent by subprocesses of {\GAP} (see "Process") is not counted.


%%%%%%%%%%%%%%%%%%%%%%%%%%%%%%%%%%%%%%%%%%%%%%%%%%%%%%%%%%%%%%%%%%%%%%%%%
\Section{Profiling}

\>ProfileOperations( true )
\)ProfileOperationsAndMethods( true )
\)ProfileMethods( <oprs> )

With these calls, the profiling is turned on.
Subsequent computations will record the time spent by each profiled
function and the number of times each function was called.
Old profiling information is cleared.

Here `ProfileOperations' turns the profiling on for all operations,
`ProfileOperationsAndMethods' for all methods of all operations,
and `ProfileMethods' for all methods of the operations in the list
<oprs>.


\>ProfileOperations( false )
\)ProfileOperationsAndMethods( false )
\)UnprofileMethods( <oprs> )

With these calls, the profiling is turned off again.
Recorded information is still kept, so you can  display it even after
turning the profiling off.

Here `ProfileOperations' turns the profiling off for all operations,
`ProfileOperationsAndMethods' for all methods of all operations,
and `UnprofileMethods' for all methods of the operations in the list
<oprs>.


\>ProfileOperations(  )
\)ProfileOperationsAndMethods(  )
\)DisplayProfile(  )
\)DisplayProfile( <funcs> )

With these calls, the collected information is displayed in the format
described below.

Here `ProfileOperations' displays only information about operations,
`ProfileOperationsAndMethods' displays information about operations
and methods.

In the first form, `DisplayProfile' displays the profiling information
for all profiled functions,
in the second form only the functions in the list <funcs> are
considered.

\begintt
gap> ProfileOperationsAndMethods(true);
gap> Factors(10^42+1);
[ 29, 101, 281, 9901, 226549, 121499449, 4458192223320340849 ]
gap> ProfileOperationsAndMethods(false);
gap> DisplayProfile();
  count  self/ms   sum/ms  chld/ms  function                                  
      1        0        0        0  DefaultRingByGenerators: method that tr*  
      1        0        0      140  Factors: for a ring element               
      1        0        0        0  ViewObj: for finite lists                 
      1        0        0        0  ForAll: method for a list/collection, a*  
      1        0        0        0  DefaultRingByGenerators                   
      1        0        0        0  ForAll                                    
      2        0        0      140  Factors                                   
      2        0        0        0  Flat: for a list                          
      2        0        0        0  Compacted                                 
      7        0        0        0  ViewObj: default method using `PrintObj'  
      8        0        0        0  ViewObj                                   
      9        0        0        0  Sort: for a mutable small list            
      9        0        0        0  Sort                                      
     10        0        0        0  Root: method for two integers             
     11        0        0        0  EQ: method for 'infinity' and 'infinity'  
     14        0        0        0  APPEND_LIST                               
     16        0        0        0  Display: generic: use Print               
     26        0        0        0  Log: method for two integers              
     29        0        0        0  Size: for a list that is a collection     
      6       10       10        0  Compacted: for a list                     
   4377       20       30       20  ADD_LIST                                  
      1      140      170        0  Factors: method for integers              
   2259      450      620      -10  POS_LIST                                  
             620                    TOTAL                                     
\endtt

% (Frank, could you describe the format?)

Note that profiling (even the command `ProfileOperationsAndMethods(true)')
can take substantial time and {\GAP} will perform much slower when profiling
than when not.


%%%%%%%%%%%%%%%%%%%%%%%%%%%%%%%%%%%%%%%%%%%%%%%%%%%%%%%%%%%%%%%%%%%%%%%%%
%%
%E

